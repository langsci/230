\author{Johanna Flick} %look no further, you can change those things right here.
\title{Die Entwicklung des Definitartikels im Althochdeutschen}  
\subtitle{Eine kognitiv-linguistische Korpusuntersuchung}
% \BackTitle{Change your backtitle in localmetadata.tex} % Change if BackTitle is different from Title

\renewcommand{\lsSeries}{eotms} % use lowercase acronym, e.g. sidl, eotms, tgdi
\renewcommand{\lsSeriesNumber}{6} %will be assigned when the book enters the proofreading stage

\BackBody{Wie in vielen anderen Sprachen der Welt hat sich auch im Deutschen der Definitartikel aus einem adnominal gebrauchten Demonstrativum herausgebildet. In der vorliegenden Arbeit wird dieser funktionale Wandel, der sich vornehmlich in der althochdeutschen Sprachperiode (750--1050 n.\,Chr.) abspielte, erstmals computergestützt und mit korpuslinguistischen Methoden anhand der fünf größten ahd. Textdenkmäler aus dem \textit{Referenzkorpus Altdeutsch} rekonstruiert. Dabei wird die Entwicklung des Definitartikels als Konstruktionalisierung der Struktur [\textit{dër}\,+\,N] begriffen: Das ursprüngliche Demonstrativum dër verliert seine zeigende Bedeutung und erschließt neue Gebrauchskontexte, in denen die eindeutige Identifizierbarkeit des Referenten auch unabhängig von der Gesprächssituation gewährleistet ist. In der Arbeit wird gezeigt, dass diese Kontextexpansion  maßgeblich von der kognitiv-linguistischen Kategorie Belebtheit beeinflusst wird.}

%\dedication{Change dedication in localmetadata.tex}
\typesetter{Johanna Flick, Sebastian Nordhoff, Felix Kopecky}
\proofreader{Andreas Hölzl,
Daniela Schroeder,
Hella Olbertz,
Katja Politt,
Lea Schaefer,
Ludger Paschen,
Mario Bisiada,
Jean Nitzke,
Sophie Ellsäßer,
Tom Bossuyt,
Yvonne Treis}

\renewcommand{\lsID}{230} % contact the coordinator for the right number
\BookDOI{10.5281/zenodo.3932780}%ask coordinator for DOI
\renewcommand{\lsISBNdigital}{978-3-96110-259-4}
\renewcommand{\lsISBNhardcover}{978-3-96110-260-0}
