\newcommand{\appref}[1]{Appendix \ref{#1}}
\newcommand{\fnref}[1]{Footnote \ref{#1}} 

\newenvironment{langscibars}{\begin{axis}[ybar,xtick=data, xticklabels from table={\mydata}{pos}, 
        width  = \textwidth,
	height = .3\textheight,
    	nodes near coords, 
	xtick=data,
	x tick label style={},  
	ymin=0,
	cycle list name=langscicolors
        ]}{\end{axis}}
        
\newcommand{\langscibar}[1]{\addplot+ table [x=i, y=#1] {\mydata};\addlegendentry{#1};}

\newcommand{\langscidata}[1]{\pgfplotstableread{#1}\mydata;}


\renewcommand{\lsCoverTitleFont}[1]{\sffamily\addfontfeatures{Scale=MatchUppercase}\fontsize{44pt}{17.25mm}\selectfont #1}

%%%%

\setcounter{tocdepth}{5}

\newcommand{\object}[1]{\textit{#1}}
\newcommand{\herkur}[1]{\textit{#1}}
\newcommand{\autor}[1]{\textsc{#1}}
\newcommand{\exref}[1]{(\ref{#1})}
\newcommand{\hervor}[1]{\glqq{#1}\grqq{}}
\newcommand{\bedeutung}[1]{\glq{#1}\grq{}}
%% TODO Wechselwirkung mit gb4e - wird das genutzt?
% \newcommand{\trans}[1]{\glqq{#1}\grqq{}}
\newcommand{\extrans}[1]{\glq{#1}\grq}
% \newcommand{\extrans}[1]{`#1'}
\newcommand{\ueberarb}[1]{\textbf{#1}}
\newcommand{\fett}[1]{\textbf{#1}}
\newcommand{\inputtable}[3]{%
  \begin{table}
      \input{#1}
      \caption{#2\label{#3}}
  \end{table}
}

%Colorbrewer YlOrRd model, for use in Abbildung 5.5
\definecolor{YlOrRd1}{cmyk}{0,0,0.30,0}
\definecolor{YlOrRd2}{cmyk}{0,0.20,0.60,0}
\definecolor{YlOrRd3}{cmyk}{0,0.45,0.70,0}
\definecolor{YlOrRd4}{cmyk}{0.05,0.77,0.80,0}
\definecolor{YlOrRd5}{cmyk}{0.25,1,0.70,0}
