\chapter{Von einer artikellosen Sprache zu einer Artikelsprache}\label{forschung}


Mit der Entwicklung des Definitartikels wandelt sich das Deutsche in wenigen Jahrhunderten von einer Nicht-Artikelsprache zu einer Artikelsprache. 
In diesem Kapitel wird dieser Wandel vor dem Hintergrund der aktuellen Forschung diskutiert. Abschnitt \ref{sec:def-ahd} geht der Frage nach, welche Möglichkeiten das Althochdeutsche kannte, um Definitheit -- trotz des Fehlens eines Definitartikels -- auszudrücken. Während sich die Forschung darüber einig ist, dass der Definitartikel seinen Ursprung im ahd. Demonstrativum \object{dër} hatte,
gibt es auf die Frage nach den Ursachen der Entwicklung ganz unterschiedliche Antworten. Diese werden in Abschnitt \ref{sec:gruende} vorgestellt und kritisch beleuchtet. Dabei haben alle Entstehungstheorien die Vorstellung gemein, dass sich der Definitartikel als Ausgleichsfunktion für bestimmte Defizite im ahd. Sprachsystem entwickelt. In Abschnitt  \ref{sec:demzudef} wird diese Auffassung problematisiert und ein Perspektivenwechsel vorgenommen: Ausgehend von der kommunikativen Leistung des ursprünglichen Demonstrativums, Referenten als diskursprominent zu markieren, wird davon augegangen, dass der Wunsch nach Expressivitiät den Wandel in Richtung Definitartikel angetrieben hat.

\section{Definitheit im Althochdeutschen} \label{sec:def-ahd}

Definitheit ist eine universelle Kategorie \parencite[269]{Leiss2000}. Sprachen, die Definitheit nicht in Form eines Definitartikels ausdrücken, nutzen alternative Strategien, um Referenten als definit zu kennzeichnen \parencite[für einen typologischen Überblick s.][]{Kramsky1972}. Das (frühe) artikellose Althochdeutsche greift ebenfalls auf verschiedene syntaktische und morphologische Mittel zurück, um definite Interpretationen zu erzielen. Sie werden in diesem Abschnitt vorgestellt.\footnote{Einführende Zusammenfassungen zum Ausdruck von Definitheit im Althochdeutschen bieten \textcite{Szczepaniak2011a,Ferraresi2014,Szczepaniak2015}.}

\subsection{Determinierer}\label{determinierer}

Neben dem sich entwickelnden Definitartikel \object{dër} verfügt das Althochdeutsche über eine Reihe von Determinierern, die zur eindeutigen Identifikation eines Referenten beitragen und somit Definitheit zum Ausdruck bringen.
Auf Basis althochdeutscher Grammatiken \parencite{, Meineke2001, Sonderegger2003,Braune2004} kann man die folgenden zwei Hauptgruppen unterscheiden:  

\begin{enumerate}[label=\alph*)]
		\item \label{poss} Possessiva: \object{m\={i}n, d\={i}n, s\={i}n} etc.
		\item \label{dieser} Demonstativa: \object{dër, dëser, jener, solih/solicher, sëlb, der samo}
\end{enumerate}

Possessiva und Demonstrativa haben gemein, dass es nicht zu ihrer Hauptfunktion gehört, Definitheit anzuzeigen, sondern sich diese quasi als Nebeneffekt aufgrund von anderen Bedeutungskomponenten ergibt: So drückt ein Possessivum typischerweise ein Besitzverhältnis aus (\object{mein Buch, meins}). Der eindeutige Bezug von Possessor und Possessum löst eine Definitheitsinterpretation aus \parencite[332]{Hoffmann2009}. Das zusammengesetzte\footnote{Es handelt sich um die Verbindung aus \object{dër} und der Partikel \object{se} (< *\object{sa}, \object{si}), vgl. \textcite[260]{Meineke2001}.} Demonstrativum \object{dëser} (nhd. \object{dieser}) wird nach \textcite[164]{Oubouzar1997a} für den unmittelbaren anaphorischen Verweis gebraucht. Die definite Lesart ist also Ergebnis der kontextuell bedingten Koreferenz. Auch die Demonstrativa \object{jener},  \object{sëlb} und \object{solih} wirken determinierend, indem sie auf etwas Vorerwähntes und somit Bekanntes verweisen.\footnote{Zur Abgrenzung von \object{solch} gegenüber anderer Determinierer s. \textcite{Demske2005}.} Sie alle haben zusätzlich das Potential, auf eine Entität in der unmittelbaren Umgebung zu referieren (vgl. den sog. situativen Gebrauch in Abschnitt \ref{sec:situativ}) und damit für eine definite Abgrenzung zu sorgen. 

\textcite[32ff.]{Lyons1999} nennt neben Possessiva und Demonstrativa für das Englische noch die   Quantifizierer \object{all, both, every} und \object{most} als adnominale Definitheitsausdrücke, deren Äquivalente auch im Althochdeutschen existieren. Während die ersten drei aus Lyons Liste jeweils auf die Totalität einer bestimmten Menge verweisen und damit die Identifizierbarkeit gewährleisten \parencite[vgl. auch][334]{Hoffmann2009}, beschreibt \object{most} ähnlich wie \object{all} eine \blockcquote[33]{Lyons1999}{proportion of some whole -- as indeed does \object{the} if it is inclusive}. Sie leisten also in bestimmten Kontexten das Gleiche wie der Definitartikel. 
Darüber hinaus kann man auch mit Ordinalzahlen einen eindeutigen Verweis erwirken, und zwar, weil die Identifikation durch eine klar definierte Position in einer bestimmten Rangordnung erfolgt, z.B. \object{der erste/zweite} aus einer bestimmten (vorerwähnten) Gruppe. Im Gegenwartsdeutschen stehen Ordinalzahlen daher in Kombination mit Definitartikel.

In Opposition zu diesen \hervor{definiten} Determinierern kennt das Althochdeutsche zudem auch artikelartige Wörter, die Indefinitheit zum Ausdruck bringen \parencite[vgl. z.B.][253f.]{Braune2004}, etwa das Indefinitpronomen \object{sum} \parencite{Donhauser2012} oder das Zahlwort \object{ein} -- der Vorläufer des heutigen Indefinitartikels \parencite{Oubouzar2000,Szczepaniak2016a}. 

Was das Stellungs- und Kombinationsverhalten betrifft, ist das Althochdeutsche  flexibler als das Gegenwartsdeutsche \parencite[vgl.][104]{Szczepaniak2011a}. So können Possessiva beispielsweise sowohl voran- als auch nachgestellt werden und auch in Kombination mit anderen Determinierern auftreten, s. \REF{ex:poss-stellung} \parencite[Beispiele aus][27f.]{Schrodt2004}.\footnote{Zum syntaktischen Status der Possessiva im Althochdeutschen s. \textcite[132ff.]{Demske2001}.} Die Studie von \textcite{Flick2018} zum ahd. Isidor deutet allerdings darauf hin, dass sowohl bei Possessiva als auch bei Demonstrativa schon im frühen Althochdeutschen die Voranstellung überwiegt.

\begin{exe}
	\ex \label{ex:poss-stellung}   
	\begin{xlist}
		\ex \label{ex:possvor} Pränominales Possessivum: \object{sinen brouder} \extrans{seinen Bruder}
		\ex \label{ex:possnach} Postnominales Possessivum: \object{chuning min} \extrans{mein König} 
		\ex \label{ex:posskombi} Kombination mit \object{dër}: \object{thaz min kind} \extrans{das mein Kind}
		\end{xlist}
\end{exe}


Für die vorliegende Arbeit ist das System der Determinierer im Althochdeutschen  vor allem aus zwei Gründen relevant: Erstens soll die Frage beantwortet werden, wie sich der emergierende definite Artikel in das System eingliedert und bestehende Distributionen verändert. Zweitens soll über die Analyse der Nominalsyntax herausgefunden werden, inwiefern sich ein Determinererschema herausgebildet hat, in welchem der Definitartikel als Teil einer neuen Konstruktion [Definitartikel + Nomen] grammatikalisiert bzw. konstruktionalisiert wird (vgl. hierzu die Ausführungen in Abschnitt \ref{sec:schema}). 

\subsection{Schwache Adjektivflexion} \label{schwache-Adjektivflexion}

Attributiv gebrauchte Adjektive stellen eine weitere Möglichkeit dar, um im Althochdeutschen Definitheit zu markieren. So können schwach deklinierte Adjektive -- in Opposition zu stark flektierten Formen -- für eine Individualisierung des Referenten sorgen \parencite[68]{Szczepaniak2011a}. Die starke Flexion bewirkt   eine indefinite Lesart, vgl. Tabelle \ref{tab:schwach-adj} und \ref{tab:stark-adj}.  Weil die schwache Deklination dem Paradigma der substantivischen \object{an-} und \object{on-}Stämme folgt, wird sie auch als nominale Deklination bezeichnet -- im Gegensatz zur pronominalen  (durch die Flexion der Pronomina beeinflussten)  starken Flexion \parencite[s.][251]{Meineke2001}. 

\begin{table}

\begin{tabular}{lllll}
\lsptoprule
                  &               & \multicolumn{1}{l}{{Maskulinum}}  & \multicolumn{1}{l}{{Neutrum}}     & \multicolumn{1}{l}{{Femininum}}       \\ \midrule
{Sg.} & {Nom.} & \textit{blinto}                          & \textit{blinta}                          & \textit{blinta}                              \\
                  & {Gen.} & \textit{blinten, -in}                    & \textit{blinten, -in}                    & \textit{blintun}                             \\
                  & {Dat.} & \textit{blinten, -in}                    & \textit{blinten, -in}                    & \textit{blintun}                             \\
                  & {Akk.} & \textit{blinton, -un}                    & \textit{blinta}                          & \textit{blintun}                             \\
{Pl.}   & {Nom.} & \textit{blinton, -un}                    & \textit{blintun, -on}                    & \textit{blintun}                             \\
                  & {Gen.} & \textit{blintono}                        & \textit{blintono}                        & \textit{blintono}                            \\
                  & {Dat.} & \textit{blintom, -on}                    & \textit{blintom, -on}                    & \textit{blintom, -on}                        \\
                  & {Akk.} & \textit{blinton, -un}                    & \textit{blintun, -on}                    & \textit{blintun}                             \\\midrule
                  & {Bsp.} & \multicolumn{1}{l}{\textit{blinto gast}} & \multicolumn{1}{l}{\textit{blinta lamb}} & \multicolumn{1}{l}{\textit{blinta kuningin}} \\
                  & {Nhd.}          & \multicolumn{1}{l}{der blinde Gast}      & \multicolumn{1}{l}{das blinde Lamm}      & \multicolumn{1}{l}{die blinde Königin}       \\ \lspbottomrule
\end{tabular}
\caption{Schwache Adjektivdeklination [+individualisierend] am Beispiel von \herkur{blinto} \extrans{blind} \parencite[226]{Braune2004}}
\label{tab:schwach-adj}
\end{table}

\begin{table}
\resizebox{\textwidth}{!}{\begin{tabular}{lllll}
\lsptoprule
{}         & {}              & \multicolumn{1}{l}{{Maskulinum}} & \multicolumn{1}{l}{{Neutrum}} & \multicolumn{1}{l}{{Femininum}} \\ \midrule
{Sg.} & {Nom.}          & \textit{blint/blinter}                  & \textit{blint/blintaz}               & \textit{blint/blint(i)u}               \\
{}         & {Gen.}          & \textit{blintes}                        & \textit{blintes}                     & \textit{blintera}                      \\
{}         & {Dat.}          & \textit{blintemu, -emo (amu)}           & \textit{blintemu, -emo (amu)}        & \textit{blintera}                      \\
{}         & {Akk.}          & \textit{blintan}                        & \textit{blint/blintaz}               & \textit{blinteru, -ero}                \\
{Pl.}   & {Nom.}          & \textit{blinte/blint}                   & \textit{blintiu/blint}               & \textit{blinto/blint}                  \\
{}         & {Gen.}          & \textit{blintero}                       & \textit{blintero}                    & \textit{blintero}                      \\
{}         & {Dat.}          & \textit{blintem, -en}                   & \textit{blintem, -en}                & \textit{blintem, -en}                  \\
{}         & {Akk.}          & \textit{blinte}                         & \textit{blint(i)u}                   & \textit{blinto}                        \\\midrule
                  & {Bsp.} & \textit{blint gast}                     & \textit{blint lamb}                  & \textit{blinto kuningin}               \\
                  & {Nhd.}          & ein blinder Gast                        & ein blindes Lamm                     & die blinde Königin                     \\ \lspbottomrule
\end{tabular}}
\caption{Starke Adjektivdeklination (a/-o-Stämme) [\textminus{}individualisierend] am Beispiel von \herkur{blint} \extrans{blind} \parencite[220]{Braune2004}}
\label{tab:stark-adj}
\end{table}


\textcite[361ff.]{Braunmuller2008} zufolge soll es sich bei der schwachen (und jüngeren) Adjektivflexion ursprünglich um ein Wortbildungsmuster gehandelt haben, das Sprecherinnen und Sprecher als Folge von Sprachkontakt als Flexiv reanalysierten \parencite[zu alternativen Entstehungsszenarien s.][13--26]{Kovari1984}. Vermutlich konnte ein schwach flektiertes Adjektiv in voralthochdeutscher Zeit selbstständig Definitheit am Bezugsnomen anzeigen 
\parencites()()[69]{Demske2001}[364]{Braunmuller2008}. Noch im ahd. Isidor finden sich Beispiele für diese selbstständige Definitheitsmarkierung, etwa \object{himiliscun got} \extrans{den himmlischen Gott} (I 7,1) \parencite[226]{Braune2004}, s. auch \textcite[69f.]{Demske2001}. 
Viel häufiger erhalten schwach deklinierte Adjektive im Althochdeutschen allerdings schon funktionale Unterstützung vom emergierendem Definitartikel, vgl. beispielhaft die Belege in \REF{ex:art-adj} aus \textcite[24,28]{Schrodt2004}, was zur einer allmählichen Routinisierung des Musters [\object{dër} + Adjektiv + Nomen] führt (s. hierzu auch Abschnitt \ref{ersatz-schwach}).

\begin{exe}
	\ex \label{ex:art-adj}   
	\begin{xlist}
		\ex \label{ex:art-adj1} \object{mit dheru smelerun dheidu} \\   \extrans{mit dem geringerem Volk} (I 9,9) 
		\ex \label{ex:art-adj2} \object{then liobon drost}  \\ \extrans{den lieben Trost} (O III 2.34)
		\end{xlist}
\end{exe}
 
Im Frühneuhochdeutschen wird der semantisch gesteuerte Verbund aus \object{dër} und schwachem Adjektiv durch eine morphologische Steuerung ersetzt. Die Adjektivflexion unterliegt dann nicht mehr der definiten oder indefiniten Lesart des nominalen Kerns, sondern reagiert auf die An- oder Abwesenheit eines Artikelwortes (=\,kooperative Flexion) \parencite[s. hierzu ausführlich][]{Demske2001,Szczepaniak2011a}. 

\subsection{Aspektoppositionen} \label{sec:aspektoppo}

Wie an den heutigen slawischen Sprachen zu sehen ist, können auch aspektuelle Oppositionen (In-)"-Definit"-heit herbeiführen: So erzielen perfektive Verben in Verbindung mit Massennomen eine definite, imperfektive dagegen eine indefinite Lesart des von der Handlung affizierten Objekts. \textcite[11ff.]{Leiss2000} illustriert diesen Definitheitseffekt am Beispiel des Russischen, s. \REF{ex:russ-aspekt}.  

\begin{exe}
	\ex \label{ex:russ-aspekt}   
	\begin{xlist}
		\ex \label{ex:russ-imper} \object{On kolol drova.} (imperfektiv) \\ \extrans{Er spaltete Holz.}
		\ex \label{ex:russ-per} \object{On razkolol drova} (perfektiv) \\ \extrans{Er spaltete das Holz.}
		\end{xlist}
\end{exe}
\noindent
Durch das imperfektive Verb \object{kolol} wird die Handlung als nicht abgeschlossen konzeptualisiert. Damit  geht die Vorstellung einher, dass Teile des Holzes noch nicht gespalten wurden, was zu einer indefiniten Lesart von \object{drova} \extrans{Holz} führt. Es wurde also nur eine unbestimmte Menge an Holz verarbeitet. Das Präfix \object{raz}- in \REF{ex:russ-per} bewirkt hingegen einen abgeschlossenen  \hervor{Außenblick} auf die Handlung. Man geht davon aus, dass das ganze Holz und damit eine abgegrenzte, definite Einheit gespalten wurde.
Das Merkmal der Totalität ist hier demnach definitorisch für die Definitheit \parencite[14]{Leiss2000}. Die Konzeptualisierungsdifferenzen gelingen am besten bei Massennomen und Objekten im Plural, da sich diese quanteln, d.h. in kleinere Teile aufspalten lassen \parencite{Heindl2016}. 

Das ahd. Verbalsystem verfügte ebenfalls über solche Perfektiv/"-Imperfektiv-Oppo"-sitionen. Allerdings ist das Aspektsystem zu Beginn der Überlieferung bereits im Abbau befindlich, so dass keine durchgängige verbale Paarigkeit wie in den slawischen Sprachen vorliegt \parencite[3]{Eroms1997}. Der Zusammenfall des Aspektsystems wird daher als ein Grund für die Herausbildung des Definitartikels vorgeschlagen (s. Abschnitt \ref{aspekt}). 


Die Perfektivierung funktioniert auch im Althochdeutschen über eine Präfigierung: Mit dem \object{gi-(ge-/ga-)}-Präfix werden imperfektive Verben in ihre perfektiven Partner transformiert. Dies bewirkt eine definite Lesart bei den Objekt-Mitspielern. Nach \textcite[176--181]{Leiss2000} kann man diesen Definitheitseffekt noch bei dem intakten Aspektpaar \object{stigan -- gistigan} im ahd. Tatian beobachten, vgl. \REF{ex:gi-stigan}. 

\begin{exe}
	\ex \label{ex:gi-stigan}   
	\begin{xlist}
		\ex \label{ex:stigan} \object{Inti sar gibot her thie iungiron \textbf{stigan} in skef}  (imperfektiv) \\ 
		\extrans{Gleich darauf forderte er die Jünger auf, in ein Schiff zu steigen.} (T 251,31--32)
		\ex \label{ex:gistigan} \object{Inti so sie tho \textbf{gistigun} in skef bilan ther uuint}  (perfektiv) \\  \extrans{Und als sie ins Schiff gestiegen waren, legte sich der Wind.} (T 255,11--12)
		\end{xlist}
\end{exe}
\noindent
So verbindet sich in \REF{ex:stigan} das imperfektive \object{stigan}  mit dem diskursneuen \object{skef}.\footnote{Es muss angemerkt werden, dass Leiss \object{skef} mit \extrans{ins Schiff} übersetzt. Dies entspricht allerdings eher einem abstrakt-situativen Definitheitskontext (vgl. Abschnitt \ref{sec:abst-sit}) als einer indefiniten Lesart. Da der Referent zuvor noch nicht eingeführt wurde, ist es jedoch passender, die Textstelle mit indefinitem \extrans{ein Schiff} zu übersetzen.} In \REF{ex:gistigan} verweist \object{skef} hingegen auf ein kurz zuvor genanntes und damit bestimmtes Boot. In diesem Fall steht es nach einer perfektiven Form \object{gistigun}.
Im direkten Vergleich zu dem russischen Beispiel aus \REF{ex:russ-aspekt} bringt Leiss' Analyse allerdings Probleme mit sich. Denn anders als bei dem Massennomen \object{Holz} ist \object{Schiff} ein singuläres und nicht teilbares Objekt, das von der Handlung \object{steigen} kaum affiziert wird. Eine nicht-totalitäre Lesart kann daher kaum entstehen: Entweder man steigt in ein Boot oder nicht. In diesem Fall liegt also keine echte (In-)Definitheitsopposition vor. 

Belege mit den Kollektiva \object{folc} \extrans{Volk} und \object{menigi} \extrans{(Menschen-)Menge} sowie mit Objekten im Plural, die Leiss an anderer Stelle zur Illustration ihrer Aspekttheorie anführt \parencite[170ff.]{Leiss2000}, sind entsprechend besser geeignet, um den postulierten Definitheitseffekt nachzuvollziehen. Denn auch hier liegt ein potentiell teilbarer Referent vor, der je nach Innen- oder Außenperspektive entweder in Teilen oder in seiner Gesamtheit vorstellbar ist, vgl. exemplarisch den Beleg in \REF{ex:menigi} \parencite[170]{Leiss2000}. 


\begin{exe}
	\ex \label{ex:menigi}  \object{gisah \textbf{trumbara} inti \textbf{menigi} sturmenta} (T 60,12)\\ 	\extrans{[Als Jesus in das Haus des Synagogenvorstehers kam und] die Flötenspieler und die Menge der klagenden Leute sah, [sagte er:]}
\end{exe}

\noindent
Das perfektive \object{gisahan} (als Pendant zu \object{sahan}) bewirkt hier die Außenperspektive auf das Verbalgeschehen. Dies führt dazu, dass sowohl die \extrans{Flötenspieler} als auch die \extrans{Menschenmenge} in ihrer Gesamtheit und damit als bestimmte, abgrenzbare Einheit interpretiert werden. 

Es gibt darüber hinaus noch ein weiteres Mittel, um im Althochdeutschen aspektuelle und damit (in-)definite Oppositionen zu schaffen: Auch alternierende Objektkasus bei perfektiven Verben sorgen für aspektuelle Differenzen  \parencite{Donhauser1990,Leiss1994,Abraham1997,Philippi1997}. Steht das Objekt im Akkusativ, erhält es eine definite Interpretation; der Genitiv erwirkt eine partitive und damit indefinite Lesart.\footnote{Diese indirekte Definitheitsmarkierung ist ebenfalls aus den heutigen slawischen Sprachen sowie dem Finnischen bekannt \parencite[74]{Philippi1997}.} Die Belege in \REF{ex:ahd-gen-akk} aus \textcite[65]{Philippi1997} sollen dies illustrieren \parencite[vgl. auch][49]{Ferraresi2014}.

\begin{exe}
	\ex \label{ex:ahd-gen-akk}
		\begin{xlist}
		\ex \label{ex:ahd-gen} \object{skanda sinan fianton \textbf{bitteres lides}}  (part. Gen./indef.) \\ 
		\extrans{Er schenkte seinen Feinden ein wenig des bitteren Getränks} (L II 53--4)
		\ex \label{ex:ahd-akk} \object{Inti dir gibu \textbf{sluzzila himilo riches}}  (Akk./def.) \\  \extrans{Und dir gebe ich die Schlüssel zum Himmelsreich.} (T 90,3)
		\end{xlist}
\end{exe}
\noindent
In \REF{ex:ahd-gen} markiert der Genitiv, dass nur ein unbestimmter Teil des bitteren Getränks (\object{bitteres lides}) eingeschenkt wurde. Der Akkusativ in \REF{ex:ahd-akk} hebt hingegen die Abgeschlossenheit der Verbalhandlung hervor, wodurch \object{sluzzila} eine definite Interpretation erhält. Kasusalternanzen gab es u.a. bei Verben der Wahrnehmung (\object{sehan, hôren}) oder Tätigkeitsverben wie \object{trinkan, ezzan, geban} \parencite[s.][100]{Donhauser1990}. Letztere wurden typischerweise mit Massennomen kombiniert und selegierten daher -- anders als heute --  meist den Genitiv \parencite[36]{Abraham1997}. Mit dem Zusammenbruch des Aspektsystems verlieren auch die Kasus ihre aspektuellen Qualitäten. Inwiefern die Entstehung des Definitartikels mit diesem Wandel zusammenhängt, wird in Abschnitt \ref{aspekt} diskutiert. 
 
\subsection{Wortstellung und Informationsstruktur} \label{is-ahd.}

Das Althochdeutsche hat eine deutlich variablere Satztopologie als das Neuhochdeutsche. Die Wortstellung ist dabei unmittelbar an informationsstrukturelle Faktoren, etwa  Topik-Kom"-mentar-Struk"-turen
\parencite{Hinterholzl2005,Ramers2005,Solf2008}, Fokus-Hin"-ter"-grund-Glie"-derungen \parencite{Petrova2009} oder an das Thema-Rhema-Gefäl"-le \parencite{Leiss2000} gekoppelt. Diese können -- ähnlich wie in anderen artikellosen Sprachen heute -- Auskunft über indefinite oder definite Lesarten von Diskursrefernten geben. Die koverte Definitheitskodierung lässt sich beispielsweise am Russischen verdeutlichen s. die Beispiele in \REF{ex:is} \parencite[s.][191]{Szczepaniak2015};  vgl.  \textcite[5]{Leiss2000} für parallele Beispiele aus dem Tschechischen. 

\begin{exe}
	\ex \label{ex:is}   
	\begin{xlist}
		\ex \label{ex:is-indef} 
		\gll \object{Na} \object{stole} \object{ležit} \object{kniga} (indefinit) \\
		auf Tisch liegt Buch\\
		\trans \extrans{Auf dem Tisch liegt ein Buch}
		\ex \label{ex:is-def} 
		\gll \object{Kniga} \object{ležit} \object{na} \object{stole} (definit) \\
		Buch liegt auf Tisch\\
		\trans \extrans{Das Buch liegt auf dem Tisch}
	\end{xlist}
\end{exe}

\noindent
Die Nachstellung von \object{kniga} in \REF{ex:is-indef} markiert, dass es sich um einen diskursneuen, d.h. rhematischen und damit indefiniten Referenten handelt. Das gleiche Subjekt ist in \REF{ex:is-def} aufgrund seiner  satzinitialen Stellung als thematisch und damit bekannt zu interpretieren. Die Voranstellung weist also auf eine definite Lesart hin.

Im Althochdeutschen spiegeln V1- und V2-Deklarativsätze ganz ähnliche (In-)"-Definit"-heits"-distributionen: In präverbaler Position stehen bekannte (definite) Referenten, die Nachstellung markiert unbekannte (indefinite) Informationen. Dies illustrieren die nachfolgenden Beispiele\footnote{Anders als in der vorliegenden Arbeit verweisen die Quellenangaben zum ahd. Tatian bei \textcite{Hinterholzl2010} auf die Edition von \textcite{Masser1994}.} aus \textcite[316]{Hinterholzl2010}, vgl. hierzu auch die Darstellung in \textcite[46f.]{Ferraresi2014}. Das finite Verb ist jeweils hervorgehoben.

\begin{exe}
	\ex \label{ex:ahd-is}   
	\begin{xlist}
		\ex \label{ex:ahd-is-indef} V1-Satz:\\
		\gll \object{\textbf{uuarun}} \object{thô} \object{hirta} \object{in} \object{thero} \object{lantskeffi}   \\
		waren dort Hirten in dieser Landschaft\\
	\trans \extrans{Dort waren Hirten in dieser Landschaft.} 
	(T 6,1)
	\ex \label{ex:ahd-is-def} V2-Satz:\\
		\gll [\object{ih} \object{bin} \object{guot} \object{hirti}]. \object{Guot} \object{hirti} \object{\textbf{tuot}} \object{sina} \object{sela} \object{furi} \object{siniu} \object{scaph} \\
		[Ich bin guter Hirte]. guter Hirte gibt seine Seele für seine Schafe\\
	\trans \extrans{Ich bin ein guter Hirte. Der gute Hirte gibt seine Seele für seine Schafe.} (T~133,11)

		\end{xlist}
\end{exe}

\noindent
In dem Präsentationssatz in \REF{ex:ahd-is-indef} erscheint das Subjekt \object{hirta} nach dem finiten Verb \object{uuarun} und damit in rhematischer Position. Dadurch erhält es eine indefinite Lesart. Im Neuhochdeutschen ist diese Wortstellung u.a. noch in Witzen konserviert \parencite[\object{Treffen sich drei Hirten...}, vgl.][83]{Ramers2005}.\footnote{Daneben kann das expletive \object{es} als Platzhalter im Vorfeld dafür sorgen, dass rhematische Informationen ans Satzende rücken, etwa \object{Es gab in dieser Landschaft Hirten} \parencite[s. z.B.][]{Hauenschild1993}.} In \REF{ex:ahd-is-def} ist \object{hirti} hingegen koreferent mit einem zuvor eingeführten Diskursteilnehmer (\object{ih bin guot hirti}) \parencite[vgl. auch][]{Solf2008}. Die definite (bzw. hier: generische) Lesart wird durch die präverbale Stellung markiert.  

Neue Diskursreferenten können im Althochdeutschen auch durch V2-Sätze eingeführt werden, wenn ein dikursgliederndes Adverbial wie \object{thô}  \extrans{da, damals} den Satz einleitet, s. \REF{ex:v2-tho} \parencite[vgl.][152]{Hinterholzl2005}. Wie im V1-Satz steht auch hier die diskursneue Konstituente postverbal. 

\begin{exe} 
\ex \label{ex:v2-tho} 
\gll \object{thar} \object{uuarun} \object{steininu} \object{uuazzarfaz} \\
	Dort waren steinerne Wasserfässer \\
	\trans \extrans{Dort waren steinerne Wasserfässer} (T 81,26)
\end{exe}

Im Laufe der ahd. Zeit etabliert sich diese historisch jüngere V2-Variante und führt -- in Verbindung mit der zunehmenden Voranstellung rhematischer, aber fokussierter Informationen -- zu einer informationsstrukturellen Neutralisierung der satzinitialen Position \parencite[s.][323]{Hinterholzl2010}. Die Entwicklung des Definitartikels verläuft parallel zu diesem syntaktischen Wandel (s. Abschnitt \ref{wortstellungswandel}). 

\subsection{Semantik und Syntax des adnominalen Genitivs} \label{sec:genitiv}

Definitheit konnte im Ahd. auch durch einen adnominalen Genitiv realisiert werden, da dieser den Referenzbereich seines Bezugsnomens eingrenzen kann und den Referenten damit identifizierbar macht. Das Beispiel in \REF{ex:genitiv} aus dem Neuhochdeutschen soll die Beziehung zwischen Bezugswort und Attribut verdeutlichen \parencite[s. auch][67f.]{Szczepaniak2011a}. 
  
\begin{exe}
	\ex \label{ex:genitiv}   
	das Gewand des Königs
\end{exe}

\noindent
Das nachgestellte Genitivattribut \object{des Königs} spezifiziert das Basissubstantiv  \object{das Gewand} und grenzt es von anderen Gewändern ab. Im Neuhochdeutschen ist die postnominale Stellung die Regel, nur Eigennamen stehen vorwiegend links des Kopfes (etwa \object{Pauls Zimmer}). Im Althochdeutschen kommen diese Attribute sowohl post- als auch pränominal vor. Eine Konstruktion mit pränominalem Genitivattribut wie in \REF{ex:genitiv-ahd} ist daher nicht markiert, sondern aufgrund ihrer hohen Frequenz im frühen Althochdeutschen sogar  \blockcquote[231]{Oubouzar1997}{die bevorzugte Position}. In \REF{ex:genitiv-ahd} bestimmt das Genitivattribut \object{chuningo} das Substantiv \object{hrucca} genauer. Es determiniert also sein Bezugswort und übernimmt damit in gewisser Weise die Funktion eines Determinativs  \parencite[236]{Oubouzar1997}.

\begin{exe}
	\ex \label{ex:genitiv-ahd}   
	 \object{chuningo hrucca}  \\
	\extrans{der Rücken des Königs} (Isidor 3,2)
\end{exe}
	
\noindent
 Angemerkt werden muss, dass die definite oder indefinite Interpretation des Kopfes sich nicht automatisch ergibt, sondern vom Kontext und der semantischen Beziehung der Konstituenten abhängt \parencite[237]{Oubouzar1997}.
Wenn der adnominale Genitiv ein Individuum bezeichnet, etwa \object{gotes sunu} \extrans{der Sohn Gottes} oder \object{Dauides burg} \extrans{Davids Stadt} liegt eine definite Lesart vor \parencite[vgl. auch][194]{Szczepaniak2015}. Auch wenn das Bezugsnomen ein funktionales Konzept repräsentiert \parencite{Lobner1985}, also eine eindeutige (nicht-ambige) Relation zu einem anderen Objekt existiert, ergibt sich eine definite Interpretation. Der adnominale Genitiv kann dann sogar einen indefiniten Referenten haben: \object{der Bürgermeister einer kleinen Stadt} \parencite[109]{Demske2001}. Bei partitiven Genitiven funktioniert dieser Definitheitseffekt nicht, vgl. das Beispiel in \REF{ex:partitiv-ahd} \parencite[s. auch][194]{Szczepaniak2015}. Während der nominale Kopf auf eine Mengen- oder Maßangabe (\object{kelih}) referiert, gibt der adnominale Genitiv an, um welche Art der Menge es sich handelt (\object{caltes uuazares}), ohne dabei zur Identifizierbarkeit des Bezugsnomens beizutragen. 

\begin{exe}
	\ex \label{ex:partitiv-ahd}   
	\object{kelih caltes uuazares}  \\
	\extrans{ein Kelch kalten Wassers} (Tatian 44,27)
\end{exe}

\section{Gründe für die Herausbildung des Definitartikels}\label{sec:gruende}

Die meisten der im vorhergehenden Teil präsentierten Strategien zur (In-)"-Definit"-heits"-markierung sind zu der Zeit, als sich der Definitartikel herausbildet, bereits im Abbau begriffen: So gehen topologische Definitheitskodierungen mit der zunehmenden Fixierung der Wortstellung allmählich verloren und indirekte Definitheitseffekte, die durch aspektuelle oder flexionsmorphologische Oppositionen entstehen, werden aufgeweicht. Daher liegt es nahe, diese Wandelprozesse mit der Entwicklung des Definitartikels in eine unmittelbare kausale Beziehung zueinander zu setzen. In diesem Abschnitt werden Forschungsmeinungen, die solche Kausalzusammenhänge annehmen, diskutiert und kritisch beleuchtet. Begonnen wird allerdings zunächst mit einer Entstehungstheorie, die einen Bereich der Grammatik berührt, der in den vorhergehenden Abschnitten noch nicht thematisiert wurde, weil er nicht mit Definitheit zusammenhängt: die nominale Flexionsmorphologie. 

\subsection{Übernahme der nominalen Flexionsendungen} \label{sec:flexion} 

Nach \textcite[168--170]{Tschirch1983} und \textcite[13]{vonPolenz2009} ist die Entwicklung des Definitartikels eine direkte Reaktion auf (schon in voralthochdeutscher Zeit eingesetzte) flexionsmorphologische Umbauprozesse: Weil der germanische Initialakzent zur steten phonologischen Reduktion der Nebensilben geführt hat, büßen die Flexionsendungen ihre lautliche Distinktivität ein und es kommt zu Synkretismen. Damit Kasus-, Numerus- und Genusinformationen dennoch eindeutig markiert werden und um Verständnisproblemen vorzubeugen, behelfen sich Sprecherinnen und Sprecher mit dem Demonstrativartikel\footnote{Laut \textcite[70]{Schildt1981} gilt diese These ebenso für den Indefinitartikel.}. Zur Illustration führt \textcite[13]{vonPolenz2009} ein Beispiel aus der \object{n}-Deklination an, die im Laufe des Althochdeutschen einen besonders starken Formzusammenfall zu verzeichnen hatte \parencite[248]{Meineke2001}, s. \REF{ex:flexion}. Eine Übersicht zu mehrdeutigen Deklinationsformen in den germanischen Sprachen bietet in diesem Zusammenhang \textcite[48--51]{Heinrichs1954}. 
 
\begin{exe}
	\ex \label{ex:flexion}   
	ahd. \object{geba, gebono, gebom} > nhd. \object{die/der/den Gaben}
\end{exe}

Die Theorie wurde allerdings vielfach kritisiert, da es durchaus Evidenzen gibt, dass Sprachen trotz stabiler Nominalflexion einen Definitartikel ausbilden. So hat sich der griechische Definitartikel bspw. zu einer Zeit entwickelt, in der das Kasussystem noch intakt war \parencite[44]{Ebert1978}. 
Auch im mit dem Althochdeutschen eng verwandten Gotischen sind die nominalen Flexionsendungen laut \textcite[10]{Kovari1984} noch so stark ausdifferenziert, dass Kasus- Genus- und Numerus eindeutig angezeigt wurden. Trotzdem zeigt das gotische Demonstrativum Züge eines emergierenden Artikels \parencite[vgl.][114--155]{Leiss2000}. Laut Oubouzar könne man zudem auch die Richtung der kausalen Korrelation anzweifeln und \blockcquote[71]{Oubouzar1992}{ebenso gut behaupten
[...], dass die Ausbildung des best. Artikels zur Abschwächung der Flexionsendungen geführt habe}; zu dieser Argumentation s. auch \textcite[51]{Heinrichs1954}.  

Die Diskussion um die Rolle des Demonstrativums als purer Marker flexionsmorphologischer Informationen hat im Bereich der ahd. Glossen eine besondere Relevanz erfahren \parencite{Glaser2000}. Mit diesen aus der Mitte des 8. Jhs. stammenden Quellen liegen die ersten Belege von [\object{dër} + N] vor; sie wurden dem lateinischen Original nachträglich  hinzugefügt. \textcite[12]{Hodler1954} zufolge sind diese Textbruchstücke als reine Übersetzungshilfe zu werten, da sie lediglich im Dienste einer schnellen Kasusindikation -- vor allem von Genitiv und Dativ -- stünden. Er degradiert das Demonstrativum in diesen Texten deswegen zum \object{Scheinartikel} \parencite[12]{Hodler1954}. Gegen diese Vorstellung wendet \textcite[209f.]{Glaser2000} ein, dass  gerade die häufigen Genitivbelege kasusmorphologisch eindeutig markiert seien. Warum sollten also diese Fälle auf eine zusätzliche Kasusindikation angewiesen sein? Zudem könne man bei den Übersetzungen häufig nicht entscheiden, ob der ahd. oder lat. Kasus angezeigt werden soll. Da die ahd. Glossen in jedem Fall immer auch von der Glossierungspraxis bzw. -absicht des jeweiligen Schreibers abhängig sind, bleibt es letztlich  -- unabhängig von der Aufgabe, die man dem Demonstrativum in diesen Schriftstücken zuschreiben will -- jedoch fragwürdig, inwiefern das Vorkommen des Demonstrativums überhaupt dem genuin-althochdeutschen Gebrauch entsprechen kann. Sicher ist, dass die Glossen keine Argumente für die These einer unmittelbaren Kausalbeziehung zwischen Flexionsschwund und Aufkommen des Definitartikels liefern.

Plausibler lässt sich der Wandel vom Demonstrativ- zu Definitartikel aus einer funktionalen Sicht erklären. Demnach nutzen Sprecherinnen und Sprecher das ahd. \object{dër}, um Referenten im Diskurs zu exponieren und als definit zu markieren, weil andere Mittel der Definitheitsmarkierung für diese Herausstellung nicht (mehr) die nötige pragmatische Stärke besitzen. Die Tatsache, dass der Demonstrativartikel darüber hinaus auch Kasus-, Numerus- und Genusinformationen transportiert, spielt allerdings für die weitere Entwicklung  der Nominalphrase seit dem Althochdeutschen eine nicht zu unterschätzende Rolle. Denn die zunehmende Obligatorisierung von \object{dër} gilt als wichtiger Meilenstein auf dem Weg zur sog. kooperativen Flexion innerhalb der Nominalphrase \parencite[s. u.a.][]{Ronneberger-Sibold2010a,Szczepaniak2010}.
 
\subsection{Ersatz für die schwache Adjektivflexion} \label{ersatz-schwach}

Wie in Abschnitt \ref{schwache-Adjektivflexion} angesprochen, stehen schwach flektierte Adjektive im Althochdeutschen häufig in Begleitung des emergierenden Artikels. Empirisch wird dies bereits im vierten Kapitel des ahd. Isidors  deutlich, s. Tabelle \ref{tab:adjektive-flick2016} aus \textcite{Flick2018}. Alle Belege mit schwach flektierten Adjektiven tragen ein \object{dër} bei sich. Im Gegensatz dazu weisen starke Adjektive bis auf eine Ausnahme keine Determination auf. Diese Korrelation kann zur Schematisierung von [\object{dër} + Adjektiv\textsubscript{schwach} + N] geführt  haben, was die Obligatorisierung von \object{dër} begünstigt (vgl. Abschnitt \ref{sec:schema}). 

\begin{table}
\centering
\begin{tabular}{lrrr}
\lsptoprule
                 & {kein Adjektiv} & {schwach flektiert} & {stark flektiert} \\ \midrule
\textit{ther}    & 31                     & 11                         & 1                        \\
kein Artikelwort & 135                    & 0                          & 15                       \\ \lspbottomrule
\end{tabular}
\caption{Adjektivflexion und Gebrauch von \object{dër} im ahd. Isidor (Kapitel 4) \parencite{Flick2018}}
\label{tab:adjektive-flick2016}
\end{table}

Die Verbindung von \object{dër} und schwach flektiertem Adjektiv ist semantisch bedingt, denn beide Elemente sorgen für eine definite bzw. individualisierende Lesart des nominalen Kerns  (vgl. die Ausführungen in Abschnitt \ref{schwache-Adjektivflexion}). Eine der Thesen zur Artikelgenese besagt daher, dass der emergierende Artikel als Kompensation zur vergleichsweise älteren schwachen Adjektivflexion entstand, weil diese nicht mehr in der Lage war, alleine die Definitheit am Kernnomen anzuzeigen \parencite{Heinrichs1954,Ebert1978,Kovari1984}. Dieser Ausgleichseffekt ließe sich \textcite[81]{Heinrichs1954} zufolge in allen germanischen Sprachen beobachten. 

In der indoeuropäischen Forschung werden unterschiedliche Gründe genannt, warum die schwache Flexion ihre individualisierende Funktion eingebüßt hat. \textcite[24]{Kovari1984} nimmt beispielsweise an, dass der Zusammenfall bestimmter Kasusendungen von schwacher und starker Adjektivflexion (insbesondere in der gesprochenen Sprache) uneindeutige Lesarten mit sich brachten, wie etwa im Gotischen: got. \object{bindans} = Akk.Pl.M, schwach und stark. Das Demonstrativum könnte in solchen Fällen zur Desambiguierung beigetragen haben. 

\textcite[44]{Ebert1978} macht daneben \parencite[mit Verweis auf][]{Kuhn1955} auch Fälle, in denen die schwache Flexionsendung zwar gesetzt wird, aber nicht automatisch zu einer definiten Interpretation geführt haben muss (etwa im Komparativ), für Inkonsequenzen in der Definitheitsmarkierung verantwortlich, welche mit dem Demonstrativum ausgeglichen werden konnte \parencite[vgl. zu dieser Argumentation auch][25]{Kovari1984}. In Bezug auf die skandinavischen Sprachen vermutet \textcite{Braunmuller2013} hingegen, dass die ursprüngliche Information [+spezifisch/definit] am Adjektiv in Sprachkontaktsituationen nicht mehr als solche erkannt wurde. 

Unabhängig davon, welche Gründe es letztlich waren, die zur Abschwächung des individualisierenden Merkmals geführt haben könnten -- die Diachronie sieht in jedem dieser theoretischen Szenarien ähnlich aus: Das ursprüngliche Demonstrativum soll als funktionale Verstärkung zu einer Adjektiv-NP bzw. einem substantivierten Adjektiv getreten sein, so dass die identifizierende Lesart des Referenten im Verbund ausgedrückt wurde. Diese Arbeitsteilung wird im Laufe der Zeit nicht mehr als solche erkannt und Sprecherinnen und Sprecher reanalysieren das Demonstrativum als alleinigen Marker für Identifizierbarkeit. Deswegen wird das Demonstrativum auch in Abwesenheit eines schwachen Adjektivs gesetzt, was zur Etablierung des Schemas [Definitartikel + N] geführt haben könnte.     

Kritisiert wurde diese Theorie vor allem deswegen, weil sie nicht für alle Sprachen Gültigkeit hat und damit keine universale Entwicklung abbilden kann. So hat der Verlust der stark-schwach-Distinktion in Sprachen wie dem Russischen keine Artikelgenese bewirkt \parencite[][64]{Philippi1997}. Allerdings verfügt das Russische (so wie die meisten slawischen Sprachen) über alternative und bis heute gut funktionierende Mittel, um Definitheit zu kennzeichnen (u.a. Wortstellung, aspektuelle Oppositionen). Vor diesem Hintergrund gab es für Sprecherinnen und Sprecher bislang wenig Anlass,  den Demonstrativartikel als neuen Definitheitsmarker einzusetzen.

Aufgrund fehlender empirischer Evidenzen, die eine direkte Abhängigkeit zwischen funktionalen Abbau im adjektivischen Flexionsparadigma und der Artikelgenese dokumentieren, muss das in diesem Abschnitt skizzierte Entstehungsszenario  zumindest angezweifelt werden, solange es einem monokausalen Erklärungsanspruch standhalten soll. Die Monokausalität selbst ist wiederum deswegen fragwürdig, weil  die schwache Adjektivflexion wohl kaum die systemische Reichweite besitzen konnte,  alle definiten NPs als solche zu kennzeichnen bzw. in Form von Substantivierungen  für individualisierte Lesarten zu sorgen. Folglich ist es unwahrscheinlich, dass der funktionale Schwund der \object{n}-haltigen Suffixe alleine eine systemische Reorganisation angekurbelt hat. 

Plausibler ist es, die stete Desemantisierung der Flexionsendung als einen von vielen Umbauprozessen zu werten, welcher dazu beigetragen hat, dass Sprecherinnen und Sprecher alternative Mittel wie den Demonstrativartikel nutzten, um schwindenden Definitheitsoppositionen entgegenzuwirken. Zu diesen Umbauprozessen gehört auch der Zusammenbruch des Aspekt"-systems, welches Thema des nächsten Abschnittes ist.

\subsection{Kompensation zum Aspektsystems} \label{aspekt}

In \ref{sec:aspektoppo} wurde gezeigt, dass aspektuelle Oppositionen (perfektive vs. imperfektive Verben, Akkusativ/Genitiv als Objektkasus) im Althochdeutschen (In-)De\-fi\-nit\-heit bewirken können. Das Aspektsystem ist allerdings zur Zeit der ersten Überlieferungen bereits im Abbau befindlich. Nach  \textcite{Leiss1994,Leiss2000,Leiss2010} springt \object{dër} ein, um dieses defizitäre System auszugleichen. Man könne dies beispielhaft im zweiten Satzteil des bereits in \REF{ex:gistigan} diskutierten Belegs beobachten \parencite[180f.]{Leiss2000}, hier noch einmal zitiert in \REF{ex:bilan}. Das Verb \object{bilinnan} gehört keinem funktionierenden Aspektpaar mehr an (*\object{linnan -- bilinnan}). Um dennoch zu kennzeichnen, dass es sich bei \object{uuint} um einen definiten Referenten handelt, wird das Artikelwort \object{ther} als Definitheitsmarker gesetzt \parencite[181]{Leiss2000}.


\begin{exe}
	\ex \label{ex:bilan} \object{Inti so sie tho gistigun in skef \textbf{bilán} ther uuint}  \\  \extrans{Und als sie ins Boot gestiegen waren, legte sich der Wind.} (T 255,11--12)
\end{exe}
\noindent
Mit dem Zusammenfall des Aspektsystems werden auch die aspektuellen Oppositionen zwischen Akkusativ und Genitiv als Objektkasus aufgeweicht. Diese funktionale Lücke wird nach \textcite[187ff.]{Leiss2000} ebenfalls durch den emergierende Definitartikel kompensiert \parencite[vgl. auch][46f.]{Abraham1997}.\footnote{\textcite[88f.]{Philippi1997} zufolge büßen die Kasus ihre aspektuelle Lesart deswegen ein, weil die Genitiv- und Akkusativendungen lautlich zusammenfallen. Dies habe die Expansion des Definitartikels im Laufe der mittelhochdeutschen Sprachperiode begünstigt; kritisch hierzu: \textcite[234f.]{Lyons1999}.}

Auch wenn diese Theorie auf den ersten Blick plausibel klingt, ist sie aus mehren Gründen problematisch. Erstens fehlen Korpusuntersuchungen, die eine Korrelation von defizitärem Aspektsystem und Artikelsetzung nachweisen. Die Umsetzung eines solchen Vorhabens ist nicht zuletzt deswegen schwierig, weil sich auf Basis der dünnen ahd. Beleglage kaum nachweisen lässt, welche Aspektpaare noch intakt sind und welche nicht.\footnote{So varrieren sogar Leiss' eigene Analysen bei ein und demselben Aspektpaar: Während \object{sehan-gisehan} auf Grundlage des in \ref{sec:aspektoppo} diskutierten Belegs (\object{gisah trumbara inti menigi...}) eine intakte Aspektopposition aufweisen soll \parencite[171]{Leiss2000}, heißt es an späterer Stelle: \blockcquote[182]{Leiss2000}{Überall dort, wo \object{gi}-Verben keine vollständig lexikalisch synonymen Aspektpartner zum Simplex darstellen, springt der definite Artikel regulär ein. So wird beispielsweise ein Verb wie \object{gi-sehan} \extrans{erblicken, wahrnehmen} zu \object{sehan} \extrans{sehen} mit dem Artikel konstruiert (\object{thó gisah
thiu menigi...})}.} Fehlende aspektuelle Partner können letztlich auch Lücken in der Überlieferung geschuldet sein. Zweitens muss auch die interne Logik der Theorie angezweifelt werden. Denn wie bereits in Abschnitt \ref{sec:aspektoppo} dargestellt wurde, ist die Reichweite des Aspekts im Satz extrem eingeschränkt. So können nur Akkusativ- und Genitivobjekte (und keine anderen syntaktischen Aktanten wie z.B. Subjekte oder Objekte im Dativ) durch das perfektive Verb als definit interpretiert werden. Damit der postulierte Definitheitseffekt überhaupt glückt, muss der Referent zudem potentiell teilbar sein. Diese Prämisse ist nur bei Massennomen oder Entitäten im Plural gegeben \parencite{Heindl2016}. Die Kompensationshypothese lässt sich damit nur auf einen sehr kleinen Teil aller Nominalphrasen anwenden. Was hingegen nicht erklärt werden kann, ist der Gebrauch von anaphorischem \object{dër} in der Subjektposition. \textcite{Leiss2000,Leiss2010} begründet diese Artikelsetzung mit dem Prinzip der Hyperdetermination.\footnote{Das Pendant hierzu ist die Hypodetermination, s. \textcite{Leiss2000}.} Darunter ist ein übergeneralisierter Gebrauch von \object{dër} zu verstehen, der eine zusätzliche (und eigentlich redundante) Markierung bereits definiter Referenten umfasst. Diese Hyperdetermination sei durch den Zusammenbruchs des Aspektsystems und daran anschießende \blockcquote[196]{Leiss2000}{Homogenisierungstendenzen innerhalb des Systems} ausgelöst worden. Eine solche Chronologie wäre aus einer gebrauchsbasierten Sicht allerdings nur dann plausibel, wenn die Kompensationskontexte, also die Belege von \object{dër} bei defektiven Aspektpaaren, so frequent wären, dass sie analogische Prozesse nach sich ziehen könnten. Dies ist im Althochdeutschen -- wie die Auswertungen von \textcite[170--181]{Leiss2000} zeigen -- gerade nicht der Fall. 
Umgekehrt tritt \object{dër} in Kontexten, die sich der Reichweite des Verbalaspekts entziehen (vor allem Subjekte oder Genitivattribute), schon vergleichsweise häufig auf \parencite[s. z.B.][]{Oubouzar1992}.  

Völlig außer Acht lässt die Aspekttheorie schließlich auch die pragmatische Kraft von \object{dër}. Zu Beginn seiner Entwicklung in Richtung Definitartikel verfügt das ursprünglichen Demonstrativum noch über eine demonstrative Komponente. Es ist unwahrscheinlich, dass \object{dër} von Anfang an nur dazu diente, die \blockcquote[281]{Leiss2000}{referentielle Qualität} eines Nomens anzuzeigen und die demonstrative Funktion keine Rolle spielte. In Abschnitt \ref{sec:kata} wird dafür plädiert, dass Sprecherinnen und Sprecher gerade diese demonstrative Funktion strategisch nutzten, um Referenten im Diskurs zu exponieren und dass dies der entscheidende Antrieb für die Herausbildung des Definitartikel ist. 

\subsection{Reaktion auf Wortstellungswandel} \label{wortstellungswandel}

In Abschnitt \ref{is-ahd.} wurde gezeigt, dass im Althochdeutschen die relativ freie Wortstellung dazu beitragen konnte, um -- vereinfacht ausgedrückt -- diskursbekannte Informationen von unbekannten syntaktisch abzugrenzen. Mit der zunehmenden Fixierung des finiten Verbs verliert dieses Verfahren an Stabilität \parencite{Hinterholzl2010}. Ob der Abbau der pragmatisch gesteuerten Wortstellung in einem unmittelbaren Zusammenhang mit der Herausbildung des Definitartikels steht, ist bislang nicht nachgewiesen. Als gesichert gilt jedoch, dass der Demonstrativartikel häufig anaphorisch gebraucht wurde \parencite{Jager1918, Oubouzar1992} und in dieser Funktion naturgemäß eine zusätzliche Markierung für thematische (also vorerwähnte) Referenten liefern konnte. Dies illustriert \textcite[161]{Leiss2000} an den ersten Zeilen des ahd. Tatians, s. \REF{ex:wort}.

\begin{exe} 
\ex \label{ex:wort} 
\glll lat.  \object{In} \object{principio} \object{erat} \object{\bf{uerbum}} \& {} \object{\bf{uerbum}} \object{erat} \object{apud} \object{deum}\\
	ahd. \object{In} \object{anaginne} \object{uuas} \object{\bf{uuort}} \object{Inti} \object{\bf{thaz}} \object{\bf{uuort}} \object{uuas} \object{mit} \object{gote} \\
	gloss. In Anfang war Wort und das Wort war mit Gott \\
	\trans nhd. \extrans{Am Anfang war das Wort und das Wort war bei Gott}  (T 1,1)
	\end{exe}

\noindent
Die anaphorische Verwendung von \object{dër} fällt in die Domäne der situationsbedingten (d.h. pragmatischen) Definitheit. Die Entwicklung zum Definitartikel beginnt allerdings erst mit der Ausweitung auf situationsunabhängige (d.h. semantische) Definitheitskontexte (s. Abschnitt \ref{sec:pragsem}), so dass man beim hier diskutierten anaphorischen Gebrauch noch nicht von einem Definitartikel sprechen kann. Ein wachsendes Bedürfnis von Sprecherinnen und Sprechern, thematische Information mit einem Demonstrativartikel als solche zu markieren, weil die Wortstellung dies nicht mehr in ausreichender Form leistete, könnte jedoch zur Routinisierung von [\object{dër} + N] beigetragen haben. In Abschnitt \ref{sec:extension} wird dieser Gedanke im Zuge der möglichen Expan\-sions\-parameter wieder aufgenommen. 

Die Wortstellung wandelt sich nicht nur auf Satz-, sondern auch auf Phrasenebene. In Bezug auf die Entwicklung des Definitartikels spielt hierbei der Stellungswandel des adnominalen Genitivs eine besondere Rolle. Abschnitt \ref{sec:genitiv} hat gezeigt, dass Genitivattribute wie in \object{der Rücken des Königs} eine definite Lesart erzeugen, wohingegen partitive Genitive eher indefinite Lesarten evozieren. Partitive Genitive stehen ursprünglich rechts ihrer Basis, nicht-partitive dagegen links der Basis \parencite[177]{Behaghel1932}. Zum Ende der althochdeutschen Sprachperiode wird die Positionsfestigkeit allerdings zunehmend aufgeweicht \parencite[235]{Oubouzar1997}. Umgekehrt rücken nicht-partitive Genitive vermehrt in die postnominale Position. Die Stellung wird damit ihrer Funktion enthoben, definite oder indefinite Lesarten zu markieren. Oubouzar geht davon aus, dass \object{dër} dazu eingesetzt wurde, um die nachgestellten Genitivtypen zu unterscheiden.  Sie verdeutlicht dies mit einem nicht-partitiven Minimalpaar, das nur wenige Zeilen voneinander im ahd. Isidor (I 9,11) zu finden ist \parencite[vgl.][79]{Oubouzar1992}.  
 

\begin{exe}
	\ex \label{ex:genitiv-stellung} \extrans{Zum Zeichen der Völker} (lat. in signum populorum)
	\begin{xlist} 
		\ex \label{ex:gen-prae} \object{in liudeo zeihne}  
		\ex \label{ex:gen-post} \object{in zeihne dhero liudeo} 		
		\end{xlist}
\end{exe}

\noindent
Während in \REF{ex:gen-prae} das Genitivattribut \object{liudeo} in pränominaler Stellung ohne Demonstrativum auftritt, erfolgt die Wiederaufnahme der Nominalphrase in \REF{ex:gen-post} mit dem Genitivattribut rechts der Basis und mit Ergänzung von \object{dhero}. Wie die Untersuchung in \textcite{Szczepaniak2015} ergeben hat, ist diese Distribution allerdings nicht regelhaft. So dokumentieren Differenzbelege (also \object{dër}-Setzungen entgegen der lateinischen Vorlage) im ahd. Isidor, dass prä- wie postnominale Genitivattribute gleichermaßen determiniert werden \parencite[199]{Szczepaniak2015}. Auch Oubouzar selbst weist an anderer Stelle  darauf hin, dass im Isidor, bei Otrid und bei Notker determinierte Genitivattribte sowohl prä- als auch postnominal vorkommen \parencite[234]{Oubouzar1997}. Was Oubouzar mit ihrem Vergleich der größten ahd. Textdenkmäler allerdings deutlich ans Licht bringt, ist die Tatsache, dass \object{dër} im Laufe der Zeit das pränominale Genitivattribut als Determinierer verdrängt \parencite[236ff.]{Oubouzar1997}. So dominiert im ahd. Isidor (um 790) der Strukturtyp [N\textsubscript{Genitiv} N] noch mit über 70\%  gegenüber dem Muster [N N\textsubscript{Genitiv}]; das Genitivattribut blockiert damit in den meisten Fällen die Verwendung von \object{dër}. Bei Notkers Boethius (um 1025) halten sich beide Strukturtypen zahlenmäßig die Waage. Wenn das Genitivattribut selbst determiniert ist und postnominal auftritt, füllt \object{dër} schon regelmäßig die pränominale Position. Dies führt zur Etablierung der heute gängigen Struktur [Det N Det\textsubscript{Genitiv} N\textsubscript{Genitiv}] wie bspw. in \object{der Hund des Nachbarn} \parencite[241]{Oubouzar1997}. 

\section{Vom Demonstrativ- zum Definitartikel} \label{sec:demzudef}

Im vorhergehenden Abschnitt wurden verschiedene Szenarien vorgestellt, die mögliche Antworten auf die Frage nach dem \hervor{Warum} der Artikelentwicklung lieferten. Die Blickrichtung ist in allen Fällen gleich: Ein ehemals funktionierendes grammatisches Teilsystem bricht zusammen, so dass ein neues Verfahren notwendig wird, das zunächst die alten Regularitäten kompensiert und dann ersetzt. Nachfolgend wird eine andere Perspektive eingenommen und nach den kognitiv-kommunikativen Beweggründen gefragt, die Sprecherinnen und Sprecher dazu bringt, das ursprüngliche Demonstrativum in neuen Kontexten zu verwenden und damit die Umfunktionalisierung in Richtung Definitartikel anzustoßen  (\ref{sec:kata}). Anschließend werden die wichtigsten Faktoren für diese Entwicklung aus der Forschungsliteratur herausgearbeitet (\ref{sec:extension}). Der letzte Abschnitt geht der Frage nach möglichen Brückenkontexten auf den Grund, also Verwendungsweisen von \object{dër}, die den Übergang vom Demonstrativ- zu Definitartikel ebnen (\ref{sec:bruecke}).  

\subsection{Expressivität als Katalysator für den Wandel} \label{sec:kata}

Die bisher genannten Entstehungsszenarien gehen davon aus, dass die Herausbildung eines Definitartikels gewissermaßen \hervor{notwendig} wurde, weil das althochdeutsche Sprachsystem nicht über die erforderlichen grammatischen Mittel verfügte, um Definitheit (oder auch Kasus- Genus- und Numerusinformationen) systematisch auszudrücken. Diese \hervor{Defizitätsperspektive} ist allerdings problematisch: Erstens suggeriert sie, dass Sprachen nur solche grammatischen Kategorien ausbilden können, die bereits existieren. Sprecherinnen und Sprecher werden dann dazu degradiert, auf Systemveränderungen lediglich zu reagieren, anstatt selbst -- angetrieben von dem Bedürfnis nach Expressivität -- in kreativer Weise Wandel anzukurbeln. Die Grammatikalisierungsforschung der letzten Jahrzehnte hat jedoch gezeigt, dass Kreativität ein wichtiger Antrieb für innovativen Sprachgebrauch ist, und dass dieser zur Etablierung neuer Kategorien führen kann, so etwa geschehen beim \object{going-to}-Futur als Tempusausdruck im Englischen \parencite[s. z.B.][30ff]{Heine1991}. Zweitens muss keineswegs erst eine Kategorie abgebaut werden, bevor Sprecherinnen und Sprecher innovativ werden, um neue Formen für alte Inhalte hervorzubringen. Als Paradebeispiel für eine solche pragmatische Verstärkung\footnote{Engl. \object{pragmatic strengthening} oder \object{pragmatic enrichment} \parencite[s.][94]{Hopper2006}.} gilt das französische Demonstrativum \object{ça}. Es ist das Ergebnis einer mehrfachen Erneuerung des ursprünglich lateinischem \object{hoc}, s. die Stadien in \REF{ex:ca} aus \textcite[94]{Hopper2006}. 

\begin{exe}
	\ex \label{ex:ca} \object{hoc} \extrans{das} > (\object{ecce}) \object{hoc}  \extrans{siehe das} > \object{eccehoc} > \object{ço} > \object{ce} > \object{ce}(\object{là}) \extrans{das da} > \object{çelà} > \object{ça}
\end{exe}

Hinter der steten formalen Erneuerung steht das Bedürfnis von Gesprächsteilnehmerinnen und -teilnehmern, sich möglichst expressiv auszudrücken \parencites[179]{Detges2002}[73]{Hopper2006}[176]{Lehmann2015}. Indem der neue Ausdruck immer routinierter gebraucht und dadurch konventionalisiert wird, geht diese anfängliche Expressivität verloren. 

In Bezug auf den Definitartikel ist sich die Forschung zwar einig, dass ein Demonstrativum am Anfang der Entwicklung steht. Wenig wurde jedoch auf dessen funktionales und expressives Potential eingegangen.\footnote{Eine wichtige Ausnahme liegt mit \textcite[16f.]{Hodler1954} vor, der das ursprüngliche Demonstrativum mit einem emphatischen Sprachgebrauch assoziiert und darin den Ursprung der Artikelentwicklung vermutet.} Was leistete \object{dër} und welchen kommunikativen Nutzen konnten Sprecherinnen und Sprecher aus dem (anfangs unüblichen) Gebrauch ziehen? Nach \textcite[40]{Lehmann2015} verfügt ein Demonstrativum neben seiner kategorialen Komponente (Pronomen oder Determinierer) über die folgenden zwei grundlegenden Bedeutungselemente:

\begin{enumerate}
\item Demonstratives Element: Definitheit und Zeigegeste (Eindeutige Referenz durch  Verweis auf Objekt)  
\item Deiktisches Element:  Distanz und Sichtbarkeit (Lokalisierung des Objekts in der Äußerungssituation)
\end{enumerate}
\noindent
Die Hauptfunktion des Demonstrativums besteht also  nicht darin, Referenten als definit, d.h. als identifizierbar zu kennzeichnen. Diese ergibt sich vielmehr indirekt und zwar, indem durch einen Verweis innerhalb einer bestimmten Äußerungssituation (oder auch: anaphorisch innerhalb eines Diskurses) die eindeutige Referenz gesichert wird. Auch vor diesem Hintergrund ist es also nicht plausibel, die ersten innovativen Verwendungen von \object{dër} lediglich als Ersatzausdruck für bestimmte Formen der Definitheit zu betrachten -- sei es als Anzeige von Abgegrenztheit/Totalität in Alternative zum perfektivem Aspekt, als Betonung für diskurs-bekannte Informationen, die durch die Wortstellung nicht mehr eindeutig hervorgehoben wird, oder in Kompensation zu schwach flektierten Adjektiven bzw. adnominalen Genitiven als Marker für individualisierte Referenten. Denn aufgrund seiner demonstrativen und lokalisierenden Kraft ist das Demonstrativum für diese Aufgaben gewissermaßen \hervor{überqualifiziert}. 

Aus einer psychologisch-kommunikativen Perspektive, wie sie explizit von \textcite{Epstein1993,Epstein1994} und \textcite{Diessel2006} eingenommen wird, besteht die Hauptaufgabe von Demonstrativa  darin, die geteilte Aufmerksamkeit von Diskursteilnehmerinnen und Diskursteilnehmern auf einen bestimmten Referenten zu lenken: \blockcquote[476]{Diessel2006}{[...] demonstratives
focus the interlocutors’ attention on a particular referent. In the
exophoric use they focus the interlocutors’ attention on concrete entities
in the physical world, and in the discourse use they focus their attention
on linguistic elements in the surrounding context. In other words, in both
uses demonstratives function to create a joint focus of attention.}

\noindent
Durch die Aufmerksamkeitslenkung führen Demonstrativa entweder ein Topik in den Diskurs ein (s. \ref{ex:topikneu}), also vereinfacht das, worüber gesprochen wird \parencite{Jacobs2001}, oder bewirken einen Topikwechsel bei mehreren bereits eingeführten Referenten (s. \ref{ex:topikwechsel}). 

\begin{exe}
	\ex 
	\begin{xlist} \label{ex:proto-dem}
		\ex \label{ex:topikneu} \object{[auf ein Messer zeigend] Gibst du mir bitte \textbf{dieses Messer}?}
		\ex \label{ex:topikwechsel} \object{Der Arzt untersuchte einen Jungen.\\ \textbf{Dieser/Dieser Junge/}?\textbf{Dieser Arzt} hatte eine besondere Krankheit.}  
		\end{xlist}
\end{exe}

\noindent
In beiden Fällen spielt die Abgrenzung zu möglichen anderen Referenten eine besondere Rolle \parencite[80]{Bisle-Muller1991}. Sie lizensiert den prototypischen Demonstrativartikelgebrauch \parencite[s. auch][]{Schlachter2015}. Referenten, die noch nicht als Topik im gemeinsamen Dikurswissen eingeführt wurden, rücken mithilfe eines Demonstrativums auf der Diskursbühne sozusagen nach vorne. Für die nachfolgende Wiederaufnahme wird dann typischerweise ein Pronomen verwendet \parencite[297]{Gundel1993}. 

Was Sprecherinnen und Sprecher erwarten, wenn sie eine mit Demonstrativum eingeleitete NP rezipieren, ist eine Neufokussierung auf einen bestimmten Diskursreferenten. Wenn allerdings Referenten, die bereits als Topik fungieren, mit einem Demonstrativum gekennzeichnet werden, erhält das Demonstrativum eine neue emphatische und expressive Wirkung: Das \hervor{Zeigen} auf zentrale und bereits mental aktivierte Referenten signalisiert, dass es sich um einen für den weiteren Diskurs besonders wichtigen Referenten handelt und dieser eine erhöhte Aufmerksamkeit verdient. Der gleiche Effekt ergibt sich für Referenten, die nicht unmittelbar eingeführt, jedoch durch das Weltwissen bekannt und einzigartig sind. Im christlichen Kulturkreis wären dies z.B. \object{der heilige Geist}, \object{die Jünger Jesus} oder \object{der Heiland}, welche im Althochdeutschen schon früh mit \object{dër} determiniert werden \parencite{Oubouzar1989}. Da keine Abgrenzungsschwierigkeiten zu anderen potentiellen Referenten vorliegt,  wäre eine normale Auszeichnung mit Demonstrativum überflüssig. Plausibler ist es in diesen Fällen, das Demonstrativum als Marker für Diskursprominenz zu betrachten. Empirische Unterstützung für diesen Ansatz kommt von \textcite{Epstein1993,Epstein1994}, der dem Altfranzösischen \object{le} ebenfalls ein expressives Potential attestiert. Es diente in der frühen Phase der Artikelentwicklung dazu, Referenten als besonders wichtig zu exponieren.\footnote{Ähnliche Beobachtungen macht auch \textcite{Laury1997} zum finnischen Demonstrativum \object{se}, welches genutzt wird, um die Aufmerksamkeit auf diskursrelevante Referenten zu lenken.} 
Ob ein Demonstrativum in diesem Sinne verwendet wird, hängt damit stark vom subjektiven Empfinden der Sprecherin bzw. des Sprechers ab \parencite[128]{Epstein1993}. Es gibt allerdings bestimmte Referenten, die prädestiniert dafür sind, die Aufmerksamkeit im Diskurs auf sich zu ziehen, nämlich menschliche, agentive und materiell gut wahrnehmbare Referenten (vgl. ausführlich Kapitel \ref{chapter:belebtheit}). Daher sind sie vermutlich Vorreiter für die Artikelexpansion und die ersten Kandidaten, die mit dem Demonstrativum in unüblichen Kontexten ausgestattet werden. Kommt es zum inflationären Gebrauch dieser Hervorhebungsstrategie, weil immer mehr Sprecherinnen und Sprecher in immer mehr Kontexten darauf zurückgreifen, schleift sich die expressive Funktion nach und nach ab \parencite[vgl. das Prinzip der unsichtbaren Hand nach][]{Keller1994}. Der Gebrauch von [Demonstrativum + N] wird zum Normalfall. Was als \hervor{Nebenprodukt} bestehen bleibt, ist die definite (identifizierende) Funktion des  Demonstrativums. Auf den damit zusammenhängenden Reanalyseprozess wird in Abschnitt \ref{sec:bruecke} näher eingegangen.  

Zusammenfassend lässt sich festhalten: Das ursprüngliche Demonstrativum wird vermutlich nicht als Verstärker oder Ersatz für Definitheit verwendet, sondern zur Hervorhebung von wichtigen Referenten und damit als Marker von Diskursprominenz. Dahinter steht das stete Bedürfnis nach Expressivität, das möglicherweise durch herkömmliche Hervorhebungsstrategien (wie etwa Wortstellung oder Akzentuierung) nicht mehr erfüllt wird. Neben der Semantik des Nomens spielen noch weitere Faktoren eine Rolle für die Expansion. Sie werden im nächsten Abschnitt erläutert.   

\subsection{Einflussfaktoren für die Entwicklung} \label{sec:extension}

Welche Faktoren determinieren den Gebrauch von \object{dër} auf dem Weg vom De"-mon"-stra"-tiv- zum Definitartikel? Die Forschung hat auf diese Frage zahlreiche Antworten geliefert und viele wurden direkt oder indirekt bereits in Abschnitt \ref{sec:gruende} angesprochen.  Was bislang  fehlt, ist eine systematische Aufstellung, die als Grundlage für weiterführende korpuslinguistische Untersuchungen dienen kann.  
Daher werden die in der Literatur diskutierten Faktoren nachfolgend fünf Gruppen zugeordnet und erläutert: Semantik des Bezugsnomens, individualisierende Attribute, syntaktische Funktion der NP, Definitheitsumgebungen im Satz und pragmatische vs. semantische Definitheitskontexte.
%\begin{description}
%\item[a] Semantik des Bezugsnomens
%\item[b] Individualisierende Attribute
%\item[c] Syntaktische Funktion der NP
%\item[d] Definitheitsumgebungen im Satz
%\item[e] Pragmatische vs. semantische Definitheitskontexte
%\end{description}
%\noindent
%Während (a) und (b) syntaktische und semantische Eigenschaften der Konstituenten innerhalb der NP betreffen, operieren die Faktoren der Gruppen (c) und (d) auf Satzebene. Gruppe (e) bezieht sich auf die Löbnerschen Definitheitsopposition, die dem kategorialen Übergang von Demonstrativ- zu Definitartikel entsprechen (vgl. ausführlich Abschnitt \ref{sec:pragsem}).
%Nachfolgend werden die einzelnen Gruppen erläutert.     

%\subsubsection{(a) Semantik des Bezugsnomens} 
Schon in den ersten Untersuchungen zum althochdeutschen \object{dër} wird das Augenmerk auf die Semantik des Substantivs gerichtet, das determiniert wird. Denn nicht alle Substantivtypen lassen sich in der Frühphase der Entwicklung gleichermaßen mit dem emergierenden Definitartikel kombinieren. So beobachtet bereits \textcite{Graf1905}, dass Eigennamen, Unika, generische Ausdrücke, Abstrakta, Stoffbezeichnungen und Kollektiva eine gewissen Resistenz gegenüber der Artikelexpansion zeigen \parencite[ähnlich][]{Bell1907, Hodler1954}. Diese Typologie reflektiert einerseits den in Abschnitt \ref{sec:gram} vorgestellten Grammatikalisierungspfad: Eigennamen und Unika erscheinen erst dann mit \object{dër}, wenn das Artikelwort seine demonstrative Bedeutung verloren hat. Generische Gattungsbezeichnungen referieren auf kein bestimmtes Individuum und bleiben deswegen länger ohne Artikel (vgl. auch Abschnitt \ref{sec-generisch} zum generischen Artikel). Dass sich auch Abstrakta, Stoffbezeichnungen und Kollektiva  gegen eine frühe Determinierung sträuben, ist ein Hinweis darauf, dass die Individualität eine Rolle für die (Nicht-)Setzung des Artikels spielt: Je weniger konturiert ein Referent ist, desto schlechter lässt er sich als identifizierbares Individuum konzeptualisieren (s. hierzu ausführlich Abschnitt \ref{sec:indi}). 
Nach Oubouzar werden Nominalgruppen, die \blockcquote[75]{Oubouzar1992}{kommunikativ besonders wichtig sind},\footnote{hier: Oubouzar 1989: S. 111 kommunikative Wichtigkeit der NG. sunu} schon im ahd. Isidor determiniert (z.B. \object{dher forasago} \extrans{der Prophet}) \parencite[vgl. auch][117f.]{Oubouzar1989}.  
Diese Aussage passt zu der Annahme, dass Sprecherinnen  und Sprecher beim frühsten (unüblichen) Gebrauch von \object{dër} das Ziel verfolgten, Referenten als diskurs"-prominent zu markieren (vgl. Abschnitt \ref{sec:kata}). 
Besonders gut geeignet für eine solche pragmatische Hervorhebung sind menschliche Referenten. Denn sie nehmen als prototypische Agens die wichtigsten Rollen im Diskurs ein und besitzen aus diesem Grund (und auch wegen ihrer materiell-konkreten Beschaffenheit) eine hohes Salienzpotential. Ihre ontologischen Eigenschaften machen Menschen zu guten Kandidaten für einen Verweis mit  Demonstrativa. Ihrer natürlichen diskursiven Relevanz ist es geschuldet, dass Sprecherinnen und Sprecher das Bedürfnis entwickeln, sie in besonderem Maße als identifizierbar zu kennzeichnen. 
Vor dem Hintergrund dieser Überlegungen kann die Hypothese aufgestellt werden, dass der Wandel vom Demonstrativ- zum Definitartikel belebtheitsgesteuert über die Grundstufen \textsc{menschlich > belebt > unbelebt > abstrakt}\footnote{Die Stufen der Artikelexpansion bei \textcite[34ff.]{Hodler1954} entsprechen im Prinzip dieser Belebtheitshierarchie.} verläuft (vgl. hierzu ausführlich Kapitel \ref{chapter:belebtheit}). Dass \object{dër} gerade bei substantivierten Adjektiven schon früh in Erscheinung tritt \parencite[44]{Ebert1978}, stützt diese Annahme. Denn den Belegen aus \textcite[55f.]{Jager1917} und \textcite[75f.]{Oubouzar1992} nach zu urteilen, handelt es sich in diesen Fällen meist um menschliche Referenten (z.B. \object{dhes almahtighin} \extrans{des Allmächtigen} \object{dher hohisto} \extrans{der Höchste}). Zusätzlich kann man davon ausgehen, dass \object{dër} auch dazu dient, den nominalen Status dieser Substantivierungen kenntlich zu machen \parencite[vgl. auch][174]{Leiss2000}. 
Nach \textcite[45]{Hodler1954} stehen zudem Substantive, die der biblischen Kultursphäre angehören, schon früh regelmäßig mit dem emergierenden Artikel (z.B. \object{ther heilant} \extrans{der Heiland}, \object{ther alteri} \extrans{der Altar} oder \object{thiu burg} \extrans{die Stadt (Jerusalem)}). Dies spricht dafür, dass eine hohe kulturelle Relevanz die Determinierung zusätzlich begünstigt (vgl. hierzu auch Abschnitt \ref{sec:relevanz}).   

%\subsubsection{(b) Individualisierende Attribute und Determinierer} 

Ob \object{dër} im Althochdeutschen gesetzt wird, hängt auch davon ab, inwiefern der nominale Kopf durch individualisierende Attribute erweitert wird \parencite[vgl.][]{Graf1905,Witzig1910,Jager1917,Hodler1954,Oubouzar1992,Oubouzar1997,Schrodt2004,Szczepaniak2015}. Es lassen sich drei Fälle unterscheiden: 

\begin{enumerate}
\item Blockierung von \object{dër}: Der pränominale Slot kann durch einen Determinierer (s. Abschnitt \ref{determinierer}) oder ein Genitivattribut (s. Abschnitt \ref{gen-attr}) belegt sein, so dass der Artikelgebrauch blockiert ist. Zwar sind Belege dokumentiert, in denen solche determinierende Attribute zusammen  mit \object{dër} auftreten \parencite[vgl. z.B. die Belegsammlungen in][60--78]{Graf1905};  allerdings zeichnet sich schon in der frühen althochdeutschen Phase die Tendenz ab, dass die NP mit nur einem Element eingeleitet wird \parencite{Oubouzar1997}.
\item Begünstigung von \object{dër}: Individualisierende Adjektive, Partizipien, restriktive Relativsätze und postnominale Genitivattribute können allesamt dafür sorgen, dass der Referenzbereich des Bezugsnomens eingeschränkt wird und eine definite Lesart entsteht. Dadurch provozieren sie die Setzung von \object{dër} \parencite[24f.]{Schrodt2004}.
\item \object{dër} bei Genitivattributen: Um ihre eigene determinierende Funktion zu stärken, werden Genitivattribute schon in den frühen Denkmälern häufig mit \object{dër} determiniert. Wie \parencite{Oubouzar1989, Oubouzar1992, Oubouzar1997} zeigt, nimmt die Determinierung im Laufe der Zeit immer weiter zu \parencite[185]{Leiss2000}. Dies begünstigt die stete Desemantisierung des ursprünglichen Demonstativs \parencite{Szczepaniak2015}.
\end{enumerate}

%\subsubsection{(c) Syntaktische Funktion der NP} 

Seiner ursprünglichen Funktion als Demonstrativum entsprechend dient \object{dër} in der Frühphase der Entwicklung vor allem dem anaphorischen Verweis \parencite[s. u.a.][]{Jager1917,Oubouzar1992,Leiss2000}. Wiederaufgenommene und damit thematische Referenten stehen typischerweise in der Subjektsposition, so dass hier ein erhöhtes Vorkommen von \object{dër} zu erwarten ist. Diese Erwartung ist auch deswegen gerechtfertigt, weil die Wortstellung im Laufe der Zeit als ikonisches Abbild der Thema-Rhema-Progression zurückgeht (vgl. Abschnitt \ref{wortstellungswandel}). Wie  \textcite[165]{Leiss2000} anmerkt, gibt es für das Ahd. noch keine Untersuchung, die eine Subjektaffinität des Artikels nachweist -- wohl aber für das Mittelhochdeutsche. So konstatiert \textcite[33ff.]{Hartmann1967} bei seiner Untersuchung rheinischer Denkmäler des Mittelalters einen erhöhten Artikelgebrauch im Subjekt im Vergleich zu anderen syntaktischen Positionen. Interessanterweise macht er das hohe Vorkommen von Personenbezeichnungen im Subjekt für diese Verteilung verantwortlich und bringt damit die Belebtheit ins Spiel. Der Belebtheitsfaktor könnte auch für das Übergewicht des Artikelgebrauchs bei Dativobjekten gegenüber Akkusativobjekten relevant sein, das Hartmann in seiner Studie herausarbeitet \parencite[42f.]{Hartmann1967}. Leiss schließt daraus: \blockcquote[165]{Leiss2000}{Da im Dativ am häufigsten semantische Rollen mit dem Merkmal [+belebt] erscheinen,
scheint dieses Merkmal, das ja auch bei Agens-Subjekten überwiegt, eine
wichtige Rolle zu spielen}. 

Bislang wurde diskutiert, welche Aktanten im Satz besonders affin gegenüber einer Auszeichnung durch \object{dër} sein könnten. Umgekehrt gibt es auch syntaktische Funktionen, die sich einer Determinierung (teils bis heute) verwehren, hierzu zählen Adverbiale, Prädikative und der Vokativ. Adverbiale liefern Hintergrundinformationen und können nicht referentiell sein (vgl. \object{zu Hause, im Frühling, mit dem Bus}). Meist geben sie in Form von Präpositionalphrasen lokale, temporale, kausale oder modale Relationen an. Wie \textcite[84]{Oubouzar1992} feststellt, kommen sie noch bei Notker, also in der späten ahd. Sprachperiode, ohne Artikelwort vor, etwa \object{in lenzen} \extrans{im Frühling}, \object{in himile} \extrans{im Himmel} oder \object{fone uuinde} \extrans{vom Winde} \parencite[vgl. auch][76]{Szczepaniak2011a}. Wie die Übersetzung zeigt, hat sich der Definitartikel in diesen Fällen auf dem Weg zum Neuhochdeutschen durchgesetzt. Sprachübergreifend konservieren adverbial gebrauchten Präpositionalphrasen (PPs) allerdings häufig die artikellose Variante \parencite{Himmelmann1998}.\footnote{Welcher Pfad eingeschlagen wird, ist \textcite[342 und 344f.]{Himmelmann1998} zufolge eine Frage von analogischen Ausgleichsprozessen und Entrenchment (vgl. Abschnitt \ref{sec:entrenchment}).} Während also eine PP mit einer spezifischen Lesart in Subjekt- oder Objektposition mit Definitartikel erscheint, bleibt sie in adverbialer Funktion artikellos, vgl. engl. \object{in winter} vs. \object{the winter is over} \parencite[332]{Himmelmann1998}. Eine ähnliche Verteilung lässt sich mit den Daten aus \textcite[84f.]{Oubouzar1992} auch fürs Althochdeutsche vermuten. Auch Prädikative stehen heute ohne Artikelwort, wenn sie auf keinen spezifischen Referenten verweisen, sondern das Subjekt als Teil einer bestimmten Klasse oder Statusgruppe ausweisen, z.B. \object{Er ist Kindergärtner/Vater/Hamburger} \parencite[vgl.][218]{DAvis2013}. Im Althochdeutschen kommen laut \textcite[6--8]{Graf1905} solche Phrasen ebenfalls meist ohne Artikelwort vor, etwa \object{thiu was witwa} (T 7,9). Ausnahmen enthalten entweder einen restriktiven Relativsatz (\object{ih bin thaz brat, thaz...} T 261,2--3) oder verweisen auf einen  vorerwähnten und damit spezifischen Referenten. Vokative blockieren ebenfalls bis heute den Artikel, da durch die direkte Anrede in der zweiten Person stets auf einen inhärent definiten Referenten verwiesen wird (z.B. \object{Frau Holle, mein Herr, liebe Kinder}). Solche Phrasen stehen, wie die Belege in \textcite[13]{Graf1905} und \textcite[40]{Bell1907} zeigen, auch im Althochdeutschen ohne Artikelwort. 

%\subsubsection{(d) Definitheitsumgebungen im Satz} 

In den vorherigen Abschnitten wurde erläutert, wie definite Referenten im Althochdeutschen mittels verschiedener morpho-syntaktischer Strategien auf Satzebene als solche gekennzeichnet werden und inwiefern dies mit dem aufkommenden Definitartikel zusammenhängen kann. Erstens wurden in Abschnitt \ref{sec:aspektoppo} Beispiele für aspektuelle Oppositionen genannt, welche  definite Lesarten evozieren: Direkte Objekte im Valenzrahmen von perfektiven Verben können als abgegrenzte und damit definite Einheit konzeptualisiert werden. Allerdings wurde gezeigt, dass diese Form der Definitheits"-kennzeichnung nur einen kleinen Teil von Nominalphrasen (Massennomen und Pluralnomen in der Funktion von Akkusativobjekten) betrifft und daher kaum als relevanter Faktor für die Artikelentwicklung in Betracht kommt. Abgesehen davon fehlen korpusgestützte Grundlagenarbeiten, die helfen, noch funktionierende von bereits defektiven Aspektpaaren abzugrenzen. Daher wird dem Aspekt als möglicher Einflussfaktor für die Artikelexpansion in der vorliegenden Untersuchung nicht weiter nachgegangen. Zweitens illustrierte Abschnitt \ref{is-ahd.}, wie die Wortstellung im Sinne von Thema-Rhema-Gliederungen als Anzeige von (In-)Definitheit fungiert. Wenn man davon ausgeht, dass die frühesten Belege von \object{dër} der zusätzlichen Hervorhebung bereits bekannter Referenten dienen (s. \ref{sec:kata}), sind thematische Umgebungen prädestiniert dafür, eine mit Demonstrativum determinierte NP zu beherbergen. Allerdings ist auch hier der empirische Aufwand relativ hoch, um eine mögliche Abhängigkeit von Wortstellung und Artikelsetzung (sowie diachrone Veränderungen in diesem Bereich) nachzuweisen. Daher wird in der vorliegenden Arbeit  dieser Punkt zu Gunsten einer detaillierten Analyse auf NP-Ebene bzw. der Funktion der NP im Satz ebenfalls ausgeklammert.

%\subsubsection{(e) Pragmatische vs. semantische Definitheitskontexte} 
Ein weiterer Gradmesser für die Artikelexpansion ist die auf \textcite{Lobner1985,Lobner1998} zurückgehende Unterscheidung von pragmatischen und semantischen Definitheitskontexten, die in Abschnitt \ref{sec:pragsem} noch ausführlich erläutert wird. Durch sie kann der kategoriale Wandel vom Demonstrativ- zu Definitartikel erfasst werden. Es wird davon ausgegangen, dass Demonstrativartikel nur in pragmatischen Definitheitskontexten gebraucht werden können, also in Kontexten, in denen der Referent nur über die unmittelbare situative oder textuelle Äußerungssituation identifiziert werden kann. Hierzu zählt der situative und anaphorische Gebrauch, der bereits oben in Beispiel \REF{ex:proto-dem} illustriert wurde, sowie der diskursdeiktische und anamnestische Gebrauch (vgl. hierzu ausführlich Abschnitt \ref{sec:demonstrativartikel}). \textcite[84--88]{Philippi1997} und \textcite[112--117]{Demske2001} attestieren dem ahd. \object{dër} genau diese Gebrauchsspanne, allerdings auf Basis von Einzelbelegen (vgl. Abschnitt \ref{sec:pragsem}). Semantische Definitheitskontexte, darunter der abstrakt-situative und der assoziativ-anaphorische Gebrauch, sind dem Definitartikel vorbehalten, vgl. zur Illustration die Beispiele in \REF{ex:sem-def}

\begin{exe}
	\ex 
	\begin{xlist} \label{ex:sem-def}
		\ex \label{ex:sem-def1} \object{Kommst du mit in \textbf{die Universtität?}} \\ Abstrakt-situativ: Eindeutige Identifikation durch Weltwissen; es gibt nur einen in Frage kommenden Referenten. 
		\ex \label{ex:sem-def2}  \object{Mein Rechner spinnt. \textbf{Die Tastatur} funktioniert nicht mehr.} \\ Assoziativ-anaphorisch: Eindeutige Identifikation durch metonymische 1:1-Be"-zie"-hung von Rechner und Tastatur
		\end{xlist}
\end{exe}
\noindent
Belege mit \object{dër}, die semantischen Definitheitskontexten wie in diesen Beispielen zuzuordnen sind, repräsentieren daher einen kategoriellen Wandel in Richtung Definitartikel. Um von einer Konstruktionalisierung des Schemas [Definitartikel + Nomen] bzw. [\object{dër} \extrans{der} + Nomen] zu sprechen, muss die semantische Definitheit allerdings als fester Bestandteil der Konstruktion konventionalisiert worden sein. Dies setzt eine gewisse Frequenz und Obligatorik solcher Fälle voraus, die bislang noch nicht nachgewiesen wurde. Beispiele für Kontexte, in denen \object{dër} eindeutig nicht demonstrativ verwendete wird, liefert \textcite[135ff.]{Kraiss2014}, s. \REF{ex:kraiss}; darunter Unika, Superlative und generische Kontexte.\footnote{Eine ausgiebige Diskussion, warum solche Kontexte in die Domäne des Definitartikels fallen, erfolgt in Kapitel \ref{chap:demdef}.}  Kraiss spürt sie auf, indem er  die größten ahd. Überlieferungen -- den Isidor, Teile des Tatians sowie Otfrids Evangelienharmonie manuell durchgeht \parencite{Kraiss2012}. An Beispielsammlungen wie diesen \parencite[vgl. z.B. auch][]{Luhr2008,Schlachter2015} knüpft die vorliegende Arbeit mit einer Korpusuntersuchung an. 

\begin{exe}
	\ex 
	\begin{xlist} \label{ex:kraiss}
		\ex \label{ex:kraiss-unika} \object{dër} bei Unika, z.B. \object{thiu erda} (O II 1.22) 
		\ex \label{ex:kraiss-superlativ} \object{dër} bei Superlativen, z.B. \object{dher hohisto} (I 4.4)
				\ex \label{ex:kraiss-generisch} \object{dër} mit generischer Referenz, z.B. \object{Thio buah duent unsih wisi} (O I 3.15)
		\end{xlist}
\end{exe}

\subsection{Brückenkontexte} \label{sec:bruecke}

Wie muss man sich den Übergang des genuinen Demonstrativums von pragmatischer zu semantischer Definitheit vorstellen? In der Forschung werden unterschiedliche Brückenkontexte (\textcite{Heine2002}, auch: \hervor{Kritische Kontexte} nach \cite{Diewald2002}) diskutiert, die den funktionalen Wandel von Demonstrativ- zu Definitartikel eingeleitet haben könnten \parencite[zur Übersicht s. auch][526--528]{deMulder2011}.

Lange Zeit wurde der anaphorische Gebrauch  \blockcquote[93]{Himmelmann1997}{mehr oder weniger fraglos} als Einfallstor für die Entwicklung gehandelt, so etwa von  \textcite{Oubouzar1989,Oubouzar1992} oder \textcite{Diessel1999}. In anaphorischer Funktion können Demonstrativa auftreten, die deiktisch neutral sind \parencite[41]{Lehmann2015}. Mit dem Abbau der deiktischen Funktion, ist es (aus semantischer Sicht) nur noch die demonstrative Kraft, die ein Demonstrativ- von einem Definitartikel unterscheidet. Nach \textcite[331ff.]{Lyons1999} können auch in situativen Gebrauchskontexten deiktische Informationen überflüssig und damit neutralisiert werden und zwar dann, wenn die unmittelbare Umgebung ausreicht, um den Referenten eindeutig zu identifizieren.\footnote{Lyons fasst daher den anaphorischen und situativen Gebrauch unter ein gemeinsames Konzept, die  \blockcquote[161]{Lyons1999}{textual-situational ostension}.} Wenn der Distanzangabe keine disambiguierende Rolle mehr zukommt, kann es leicht zur funktionalen Überlappung von Demonstrativ- und Definitartikel kommen, wie etwa in den folgenden Beispielen aus dem Englischen \parencite[164]{Lyons1999}. 


 \begin{exe}
	\ex 
	\begin{xlist} \label{ex:lyons}
		\ex \label{ex:lyons-sit}  \object{[In a room where there is just one stool] Pass me \textbf{the/that stool}, please.}  
		\ex \label{ex:lyons-ana} \object{The exam results came out this morning, and \textbf{the/those students} who passed are already at the pub celebrating.}  
		\end{xlist}
\end{exe}

\noindent
Nach \textcite[332]{Lyons1999} können solche Kontexte die Deakzentualisierung und weitere lautliche Reduktion des Demonstrativums  initialisieren -- es kommt zur Renanalyse (vgl. Abschnitt \ref{sec:reanalyse}). Von da aus kann der Artikel auf neue Kontexte übertragen werden. 

Einwände gegen Lyons Modell bzw. die allgemeine Hypothese, dass der anaphorische Gebrauch den Ausgangspunkt für den funktionalen Wandel darstellt, werden in 
\textcite[527]{deMulder2011} auf Basis der Argumentation von \textcite[96--98]{Himmelmann1997} erbracht. So gäbe es Sprachen, die über Demonstrativa verfügen, welche primär für den anaphorischen Gebrauch bestimmt sind (z.B. lat. \object{is, ea, id}). Doch gerade für diesen Typus lässt sich -- folgt man \textcite[98]{Himmelmann1997} -- keine Entwicklung in Richtung Definitartikel dokumentierten.  Zudem fungieren aus typologischer Sicht meist die ferndeiktischen Demonstrativa als Quelle, wie etwa das distale lat. \object{ille}, aus dem sich die Artikel in den romanischen Sprachen entwickelt haben. Der Distanzparameter scheint damit -- anders als in Lyons Szenario postuliert -- doch eine Rolle bei der Entwicklung zu spielen. Interessanterweise sind beide Argumente auf den deutschen Definitartikel nur schlecht übertragbar: Denn das ahd. \object{dër} diente häufig der anaphorischen Wiederaufnahme  \parencite[z.B.][]{Jager1917, Jager1918,Oubouzar1992,Leiss2000}. Es könnte sich von hier also ein direkter Pfad in Richtung Definitartikel ansetzen lassen. Zweitens handelt es sich bei \object{dër} gerade nicht um ein distales, sondern um ein proximales bzw. distanzneutrales Demonstrativum \extrans{dieser}. 


Himmelmanns stärkster Einwand gegen die \hervor{Anapher-Hypothese} zielt auf die bis dato kaum diskutierte Frage, wie der Übergang von situationsabhängigen (pragmatischen) zu situationsunabhängigen (semantischen) Gebrauchskontexten modelliert werden kann: \blockcquote[94]{Himmelmann1997}{Es ist nicht einfach zu erkennen, wo ein Anknüpfungs- oder Übergangspunkt zwischen dem anaphorischem Gebrauch und den für Definitartikel definitorischen, semantisch definiten Gebrauchskontexten gegeben ist. Die ähnliche Bezeichnung von anaphorischem und assoziativ-anaphorischem Gebrauch suggeriert, daß dies engverwandte Gebrauchsweisen sind und deshalb auch kein besonderer Erklärungsbedarf besteht. Aber das ist m.E. nicht der Fall, denn bei asso"-ziativ-ana"-phorischem Gebrauch findet nicht eigentlich ein Rückverweis auf eine
vorangehende Nennung statt. Vielmehr wird durch die vorangehende Nennung eine Art Rahmen geschaffen, in dem gewisse Referenten vorgesehen sind. 
Es wäre also zu zeigen, wie der Übergang vom Verweis auf einen
im vorangehenden Diskurs erwähnten Partizipanten (=\,anaphorischer Gebrauch) auf einen
nicht vorerwähnten, im Redeuniversum latent 'vorhandenen' Referenten (=\,assoziativ-anaphorischer Gebrauch) zu konzipieren ist.}

Die Lösung sieht \textcite{Himmelmann1997} im anamnestischen Gebrauch oder engl. \object{recognitional use} (s. hierzu ausführlich Abschnitt \ref{sec:amnamnestisch}). Hierbei wird das Artikelwort genutzt, um den Adressaten an einen spezifischen Referenten aus dem gemeinsamen Wissensrahmen zu erinnern \parencite[61 und 81]{Himmelmann1997}, vgl. die Beispiele in \REF{ex:himmelmann}. Um den Referenten zu identifizieren, wird  entweder auf ein früheres Gespräch Bezug genommen (wie in \ref{ex:anamnestisch-party}) oder -- mit Hilfe beschreibender Attribute --  an einen möglichen gemeinsamen Erfahrungshorizont appelliert (wie in \ref{ex:anamnestisch-matcha}).  

 \begin{exe}
	\ex 
	\begin{xlist} \label{ex:himmelmann}
		\ex \label{ex:anamnestisch-party}  \object{Warst du am Samstag noch auf \textbf{dieser/der Party}? [über die wir gesprochen hatten]}
		\ex \label{ex:anamnestisch-matcha} \object{Ich habe heute \textbf{dieses/das neue Getränk mit Matcha} ausprobiert, das man jetzt in der Mensa kaufen kann.}  
		\end{xlist}
\end{exe}

\noindent
Weil auch bei semantisch-definiten Gebrauchskontexten auf gemeinsames Wissen Bezug genommen wird, kann man hier eine konzeptuelle Nähe zum anamnestischen Gebrauch ansetzen. Der Unterschied besteht lediglich 
in der Größe des Kreises an Mitwisserinnen und Mitwissern: Während beim anamnestischen Gebrauch wenige \hervor{eingeweihte} Diskursteilnehmer über das Wissen, mit dem der jeweilige Referent zu identifizieren ist, verfügen, wird sowohl beim abstrakt-situativen als auch beim assoziativ-anaphorischen Gebrauch (vgl. die Beispiele oben in \REF{ex:sem-def} und ausführlich die Abschnitte \ref{sec:abst-sit} und  \ref{sec:asso}) Bezug auf das allgemeine Weltwissen genommen, welches die Sprechergemeinschaft teilt \parencite[95]{Himmelmann1997}. Für das Althochdeutsche lassen sich insbesondere bei Referenten, die den christlich geprägten Adressaten bekannt sein sollten, anamnestisch verwendete \object{dër}-Phrasen belegen, etwa das nicht vorerwähnte \object{dhiu magad},  das auf die Jungfrau Maria verweist \parencite[74]{Szczepaniak2011a} oder \object{dhiu burc}, das immer die Stadt Jerusalem meint \parencite{Flick2018}.\footnote{Beide sind im ahd. Isidor belegt.} Problematisch könnte an diesem Übergangsszenario sein, dass anamnestische Gebrauchskontexte womöglich nicht die notwendige Frequenz erlangen, um alleine für die Reanalyse des Demonstrativums zum Definitartikel zu sorgen. 

Eine möglich \hervor{Rettung} für den anaphorischen Gebrauch als Brückenkontext unternimmt  
\textcite{Schlachter2015}. Auf Basis von \textcite{Bisle-Muller1991} betrachtet sie das Abgrenzungsmerkmal als entscheidendes Merkmal vom Demonstrativa: Ein Demonstrativum signalisiert immer, dass andere potentielle Referenten im Spiel sein könnten und zwar unabhängig davon, ob diese explizit zur Auswahl stehen oder nicht (vgl. zu diesem Ansatz auch  Abschnitt \ref{sec:kata}). In Kontexten, in denen die Abgrenzungsinformation redundant ist, kann es zur Reanalyse kommen, da der Adressat im Sinne der Grice'schen Quantitätsmaxime \parencite[Mache deinen Beitrag nicht informativer als notwendig,][]{Grice1975} dazu eingeladen wird, das Abgrenzungsmerkmal zu streichen \parencite{Schlachter2015}. Dieses Szenario lässt sich leicht auch auf den anamnestischen Gebrauch übertragen \parencite[s. hierzu][]{Schlachter2015}, so dass man von mindestens zwei Brückenkontexten ausgehen muss. Zu einem ähnlichen Schluss kommt auch \textcite{Himmelmann1997}: 

\blockcquote[96]{Himmelmann1997}{
Ganz generell scheint es mir verfehlt, nach genau einem, vermeintlich entscheidenden
Übergangspunkt bei der Grammatikalisierung von Definitartikeln zu suchen. Eine vollständige
Theorie der Artikelgenese müßte vielmehr ein ganzes Netz von möglichen Übergangspunkten
zwischen pragmatisch-definiten und semantisch-definiten und zwischen verschiedenen
se"-man"-tisch"=de"-fi"-ni"-ten Gebrauchskontexten aufweisen.} 

\noindent
Ein Ziel der vorliegende Arbeit ist es, diesem Netz von möglichen Brückenkontexten im Althochdeutschen auf den Grund zu gehen. 

\section{Zusammenfassung}
Definite Referenten konnten im Althochdeutschen über eine Reihe unterschiedlicher morphosyntaktischer Strategien markiert werden. Zum einen sorgten -- ähnlich wie im Gegenwartsdeutschen -- Phraseneinleiter wie Possesssiv- und Demonstrativartikel sowie Genitivattribute dafür, dass das Bezugsnomen eindeutig identifizierbar war. Andere Mittel wurden im Laufe der Zeit abgebaut, darunter individualisierende Adjektive und aspektuelle Oppositionen sowie die ehemals freie Wortstellung, die pragmatisch genutzt wurde, um definite Referenten syntaktisch zu exponieren. In diesem Kapitel wurden Forschungsmeinungen diskutiert, die die Entwicklung des Definitartikels als Reaktion auf diese systemischen Umbrüche begreifen. Es ist deutlich geworden, dass diese meist monokausal ausgerichteten Erklärungsansätze jeweils nur eine bestimmte Menge an \object{dër}-Setzungen erklären können und dabei die kommunikative Funktion des genuinen Demonstrativums zu sehr vernachlässigen. Vor diesem Hintergrund wurde ein alternatives Entstehungsszenario präsentiert: Zugunsten einer hohen Expressivität nutzen Sprecherinnen und Sprecher das Demonstrativum, um besonders wichtige Referenten im Diskurs hervorzuheben. Ähnlich wie bei anderen Grammatikalisierungsprozessen (z.B. dem Negationswandel) kommt es zur Inflation dieser ehemals innovativen Strategie, was eine semantische Ausbleichung und zunehmende Obligatorisierung der Struktur [\object{dër} + N] mit sich bringt. Ob ein Substantiv mit dem emergierenden Artikel gekennzeichnet wird, ist vor allem eine Frage seiner Semantik -- und wie in Kapitel \ref{chapter:belebtheit} noch gezeigt wird -- abhängig von seinem Belebtheitsgrad. Darüber hinaus wurden weitere  Faktoren, die in der Forschung als Determinanten für den Wandel bzw. die \object{dër}-Setzung betrachtet werden, systematisch geordnet. Im letzten Abschnitt erfolgte eine Diskussion möglicher Brückenkontexte, wobei festgehalten werden kann, dass sowohl der anaphorische als auch der anamnestische Gebrauch für den konzeptuellen Übergang zu situationsunabhängigen Definitheitslesarten plausibel erscheint. 
