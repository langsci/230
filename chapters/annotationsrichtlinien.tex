\chapter{Anhang: Annotationsrichtlinien}
\section{Belebtheit} \label{sec:richtlinien-belebtheit}

%Belebtheit wird bei Substantiven mit den folgenden POS-Tags annotiert: \\ADJOS, ADJS, VVPPS, VVPSS, NA, NE, NEO.

Bei der Belebtheit handelt es sich um eine kognitiv-linguistische Kategorisierung. Sie bezieht sich auf Nomen bzw. ihre Referenten, also das, worauf sich ein Nomen in der außersprachlichen Welt bezieht. Die einfachste Art, um Referenten nach Belebhteit zu kategorisieren, ist die Unterscheidung in \object{belebt} (z.B. \object{Mutter, Jünger}) und \object{unbelebt} ({z.B. \object{Haus, Stein}). Für die vorliegende Annotationsaufgabe sind die Belebtheitsabstufungen aber noch feiner differenziert. 

Nachfolgend sieht man eine Übersicht über die zu vergebenen \object{Tags}, d.h. die Kategorien, nach denen ein Lemma gekennzeichnet, also annotiert werden kann. Jedes Lemma kann genau einen \object{Tag} haben (zu uneindeutige Fällen s.u.). Sie werden immer klein geschrieben und sind mit Unterstrich getrennt; es gibt keine Umlaute. Jede Kategorie wird auf den nächsten Seiten genauer mit Beispielen erklärt.

\vspace{-3ex}
\noindent\parbox[t]{2.4in}{\raggedright%
\begin{itemize}
\setlength{\itemsep}{0pt}
\item Übermenschliche Referenten
  \begin{itemize}
    \setlength{\itemsep}{-5pt}
  \item ueber\_pos
  \item ueber\_neg
  \end{itemize}
\end{itemize}
}%
\hfill%
\parbox[t]{2.4in}{\raggedright%
\begin{itemize}
\item Konkreta (belebt)
  \begin{itemize}
    \setlength{\itemsep}{-5pt}
  \item kon\_mensch
  \item kon\_tier
  \end{itemize}
\end{itemize}
}

\vspace{-3ex}
\noindent\parbox[t]{2.4in}{\raggedright%
\begin{itemize}
\item Konkreta (unbelebt)
  \begin{itemize}
    \setlength{\itemsep}{-5pt}
  \item kon
  \item kon\_koerper
  \item kon\_ort
  \end{itemize}
\end{itemize} 
}%
\hfill%
\parbox[t]{2.4in}{\raggedright%
\begin{itemize}
\item Abstrakta
  \begin{itemize}
    \setlength{\itemsep}{-5pt}
  \item abst
  \item abst\_kon
  \end{itemize}
\end{itemize}
}


Wenn ein Referent nicht eindeutig in eine dieser Gruppen passt oder anderweitig Probleme bereitet, dann wird er als  \object{unklar} gekennzeichnet. Die in Frage kommenden Kategorien werden mit einem Unterstrich daran gehängt. Besteht bspw.\thinspace{}Unsicherheit, ob es sich bei einem Referenten um \object{kon\_mensch} oder \object{ueber\_pos} handelt, kennzeichnet man dies so: 

\begin{itemize}
\item unklar\_kon\_mensch/ueber\_pos
\end{itemize}

\noindent 
Wichtig: Die Reihenfolge, der in Frage kommenden Kategorien, muss alphabetisch sein. Wenn die Wortbedeutung unklar ist, dann wird dies folgendermaßen gekennzeichnet: 
\begin{itemize}
\item unklar\_wort
\end{itemize}


\subsection{Übermenschliche Referenten}\label{ueber}

In diese Kategorie fallen biblisch aufgeladene Referenten mit übersinnlichen Fähigkeiten wie \object{Gott, Jesus, der (heilige) Geist, Teufel, Engel}. Wir unterscheiden positive und negative Referenten. 

\begin{itemize}
\item ueber\_pos \\
z.B.\thinspace{}\object{Gott, Engel, Geist, Heiland}
\item ueber\_neg \\
z.B.\thinspace{}\object{Teufel, Beelzebub, Antichrist, Dämon}
\end{itemize}


\subsection{Konkreta (belebt)}

Diese Kategorie umfasst belebte Referenten, also Menschen oder Tiere: Sie sind
irdisch und handlungsfähig. Auch wenn Menschen oder Tiere in einer Gruppe
auftreten (\object{Volk, Herde}) oder ihre Bezeichnung eine gewisse Abstraktion
beinhaltet (wie z.B. der auf bestimmten Verwandtschaftsbeziehungen beruhende
Begriff \object{Familie}), fallen sie in diese Kategorie. (Für die Auswertung ist in erster Linie wichtig, Menschen und Tiere von unbelebten Referenten zu unterscheiden.)


\begin{itemize}
\item kon\_mensch \\
z.B. \textit{Knecht, Prophet, Angeklagter, Tochter, Familie, Volk, Maria, Apostel, Herr}
\item kon\_tier \\
z.B. \textit{Hund, Wolf, Schaf, Stier, Ameise, Vogel, Herde}
\end{itemize}

\subsection{Konkreta (unbelebt)}

In diese Gruppe fallen Referenten, die physisch fassbar sind und die man (theoretisch) ansehen kann, die aber nicht belebt sind. Auch Pflanzen gehören in diese Gruppe.

\begin{itemize}
\item kon \\
Konkrete Referenten, z.B. \textit{Tisch, Stein, Blatt, Gold, Baum}
\item kon\_koerper \\
Körperteile, d.h. Teile, die fest zu einem (belebtem) Körper gehören und nicht \hervor{veräußerbar} sind, z.B. \textit{Kopf, Arm, Auge, Haar, Blut}
\item kon\_ort \\
Potentielle (reale) Aufenthaltsorte für Menschen\footnote{\textcite[s.][Abschnitt zur Belebtheitsannotation]{Garretson2010}}, z.B. \textit{Platz, Wiese, Stadt, Straße, Küche, Haus}
\end{itemize}

\subsection{Abstrakta}

Abstrakta bezeichnen etwas Nichtgegenständliches und sind für unsere Sinne nicht direkt wahrnehmbar. Typische Abstrakta wie \object{Friede, Hoffnung, Neid, Trauer, Idee} kann man daher nicht anfassen oder sehen. Sie beruhen auf einem Abstraktionsprozess und sind etwas vom Menschen Erdachtes. Man kann sich keinen konkreten und klar umrissenen Referenten vorstellen  \parencite[zur Vertiefung s. ][]{Schrauf2011,Ewald1992}.

Nachfolgend sind Beispiele für diese Kategorie aufgeführt.\footnote{Die Liste entstammt \textcite[143]{Schrauf2011}. Die Gruppe der \hervor{Menschlichen Vorstellungen} wurde um \object{Gott, Geist, Teufel, Fee, Gespenst, Monster} gekürzt, da diese gesondert betrachtet werden s. Abschnitt \ref{ueber}. Zudem wurden belebte Referenten, die typischerweise als Agens fungieren (\object{Boss, Chor, Chrew, Dozent, Ehefrau, Enkel, Feind, Geschwister, Herr, Paar}),  aus der Gruppe \hervor{Verhältnisse} entfernt; sie werden hier als \object{kon\_mensch} gekennzeichnet.} Die Beispiele dienen lediglich zur Veranschaulichung: Bei der Annotation werden alle Referenten, die in eine dieser Kategorien zugeordnet werden könnten, gleichermaßen mit \object{abst} gekennzeichnet. 
%(Später kann eine genauere Ausdifferenzierung dieser Gruppe erfolgen.)

\fett{Handlungen}: \object{Anklage, Ausführung, Befragung, Betonung, Betrachtung, Diebstahl, Fluchen, Gebärde, Jubel, Kampf, Krieg, Kuss, Lüge, Mord, Prahlerei, Schrei, Singen, Spaziergang, Spiel, Sport, Streik, Tanken, Tanzen, Vortrag}

\fett{Vorgänge}: \object{Alterung, Begegnung, Blitz, Blühen, Denken, Donner, Ebbe, Eingebung, Entstehung, Entwicklung, Erdbeben, Feier, Flut, Gewitter, Knall, Leben, Schall, Schlaf, Sturz, Traum, Untergang, Verlust, Welken, Wind}

\fett{Verhältnisse}: \object{Ähnlichkeit, Bande, Beziehung, Brut, Ehe, Eintracht, Entsprechung, Lobby, Macht, Pendant, Relation, Vergleich, Verhältnis, Verwandtschaft}

\fett{Maße und Zeiten}: \object{Abmessung, Ära, Breite, Datum, Dosis, Durchmesser, Fortdauer, Frist, Gramm, Jahreszeit, Kilogramm, Liter, Maß, Maßeinheit, Meile, Nacht, Ohm, Quote, Start, Watt, Winter, Wochentag, Zeit, Zeitalter}

\fett{Zustände}: \object{Armut, Bann, Befinden, Chaos, Depression, Durst, Erschöpfung, Existenz, Gesundheit, Hast, Helligkeit, Hitze, Hunger, Krise, Lärm, Müdigkeit, Rausch, Rheuma, Ruhe, Stille, Stress, Sucht, Trance}

\fett{Eigenschaften}: \object{Duft, Ehre, Ehrlichkeit, Einfachheit, Fähigkeit, Farbe, Fleiß, Freundlichkeit, Geiz, Größe, Humor, Klang, Kraft, Leichtigkeit, Menschlichkeit, Mut, Offenheit, Prunk, Qualität, Sorgfalt, Stärke, Stil, Tugend}

\fett{Wissenschaften}: \object{Chemie, Code, Didaktik, Differenz, Formel, Funktion, Gen, Inflation, Ion, Konvergenz, Kunst, Matrix, Monopol, Physik, Produktion, Rasse, Reim, Statistik, Stern, Technik, These, Vers, Wissenschaft, Zins}

\fett{Emotionen}: \object{Angst, Ärger, Ekel, Empörung, Glück, Jammer, Leidenschaft, Liebe, Neid, Qual, Reue, Schamgefühl, Scheu, Schmerz, Sicherheit, Sorge, Vertrauen, Verwirrung, Wohlgefühl, Wonne, Wut, Zorn, Zuneigung, Zweifel}

\fett{Menschliche Vorstellungen}: \object{Bewusstsein, Einbildung, Gewissen, Hölle, Idee, Illusion, Jenseits, Kult, Seele, Selbst, Sinn, Spuk, Trick, Unterwelt, Vision, Vorstellung, Wirklichkeit, Zauberei} \\

Substantive, die sowohl abstrakt als auch konkret interpretierbar sind, werden als \object{abst\_kon} gekennzeichnet, etwa \object{Vertrag, (Vor-)Wort, Himmel, Erde, Stimme}. 
Diese Substantive erkennt man daran, dass entweder typische Kontexte denkbar sind, die dazu führen, dass der Referent unmittelbar sichtbar (z.B. \object{Himmel}), greifbar (z.B. \object{Vertrag}) oder hörbar (z.B. \object{Stimme}) ist und damit als etwas Konkretes betrachtet werden kann. Oder man sieht, dass eine gewisse Abstraktion involviert ist: So wäre eine \object{Hose} bspw. \object{konkret}, aber die \object{Kleidung} \object{abst\_kon}. Hier sieht man die genannten Kategorien mit Beispielen im Überblick:   

\begin{itemize}
\item abst \\
typische Abstrakta, z.B. \textit{Glück, Neid, Friede, Ehe, Begegnung, Ungerechtigkeit, Leben, Tod}
\item abst\_kon \\
untypische Abstrakta, z.B. \textit{Wort, Himmel, Stimme, Vertrag, Ernte}
\end{itemize}

\subsection{Hinweise zu den Übersetzungen im DDD}

Bei der Vergabe der Belebtheitskategorien soll immer die ganze Beschreibung pro Zeile in Betracht gezogen werden, d.h.\thinspace{}z.B. 

\begin{itemize}
\item ein \object{Bote} wäre \object{kon\_mensch} aber \object{Bote, Herold, (Ab)gesandter, Apostel, Engel} wäre quasi ein göttlicher Bote und daher \object{ueber\_pos}; 
\item die \object{Elle} alleine wäre \object{kon\_koerper}, aber \object{Elle [Längenmaß]} wäre eine Maßangabe und damit \object{abst}.
\end{itemize}

Meistens handelt es sich bei der Auflistung in der DDD-Liste aber um bedeutungsähnliche Begriffe, wie bei den folgenden Beispielen:

\begin{itemize}
  \itemsep0pt
\item \object{Not(lage), Zwang, Gewalt}
\item \object{Not(lage), Zwang, Gewalt, Bedrängnis, Drangsal}
\item \object{Not(lage), Zwang, Gewalt, Bedrängnis, Drangsal, Bedarf, Bedürfnis,...} 
\end{itemize}

→ 3 Mal \object{abst}

Manchmal kommt es vor, dass die Aufzählung mehrere unterschiedliche Kategorien abdeckt, so dass kein einheitliches Konzept vorliegt. Solche Fälle werden dann mit  \object{unklar} + der Angabe der jeweiligen Kategorien gekennzeichnet, z.B.

\begin{itemize}
\item \object{Einsamkeit, Verlassenheit, Einöde, Wüste}
\end{itemize} 

→ \object{unklar\_abst/kon\_ort} (Reihenfolge der Kategorien bitte alphabetisch)

\noindent 
\fett{Aufgepasst:} Wenn die Aufzählung konkrete und abstrakte Dinge enthält (dies ist recht häufig der Fall), wird \object{abst\_kon} annotiert, z.B. 

\begin{itemize}
  \itemsep0pt
\item \object{Erscheinung, Angesicht, Antlitz}
\item \object{Ende, Gipfel, Spitze, Wipfel}
\item \object{Ertrag, Lohn}
\end{itemize}
→ jedes Mal \object{abst\_kon} 
\\
\fett{Faustregel:} Wenn bei einer oder mehreren Bezeichnungen nicht eindeutig festgestellt werden kann, ob es sich um etwas Konkretes oder Abstraktes handelt, dann immer \object{abst\_kon} vergeben. 

\section{Ausprägungen der Variable \hervor{Definitheit}} \label{sec:richtlinien-definitheit}
 
Definitheit wird in der Spalte \hervor{np.definit} annotiert. Eine tabellarische Übersicht der Kategorien ist im Hauptteil der Arbeit in Abschnitt \ref{sec:annotationsschritte} zu finden. Nachfolgend werden die einzelnen Kategorien detailliert erläutert. 
 

\subsection{Situativer Bezug}

Der Verweis der NP gilt einem Referenten in der außersprachlichen Wirklichkeit, z.B.: 

\begin{enumerate}
\item Gibst du mir bitte diesen Stift? 
\item Achtung, dieser Hund beißt. 
\item Diese Stadt gefällt mir nicht.  
\end{enumerate}

\noindent 
\fett{Hinweis zur Identifikation:} Zeigegeste denkbar 

\subsection{Anaphorischer Bezug}

Die NP ist koreferentiell mit einem unmittelbar zuvor genannten Referenten, z.B.: 

\begin{enumerate}
\item Dort ist eine Stadt. Diese Stadt...
\item Teil 1 beschreibt die Dreieinigkeit. Auch in Teil 2 wird diese Dreieinigkeit beschrieben.
\item Der Prophet erzählte uns, dass... Der/Dieser Prophet sagt uns auch, dass...
\end{enumerate}

\noindent 
\fett{Hinweis zur Identifikation:} Es muss eine NP existieren, die in der unmittelbaren Äußerung, d.h. 1-3 Sätze davor, bereits eingeführt wurde. Der Referent ist also bereits im Kurzzeitgedächtnis aktiviert. Die Koreferenz muss daher gekennzeichnet werden. 

\noindent 
\fett{Kennzeichnung der Koreferenz:} In der Spalte \object{Koreferenz} wird die ID zum Kopf der koreferenten NP angegeben. 

\subsection{Diskursdeiktischer Bezug}

Die NP verweist (zusammenfassend und/oder evaluierend) auf größere syntaktische Einheiten, also Sätze, Abschnitte oder Texte, indem ein neuer Referenten eingeführt wird, z.B.: 

\begin{enumerate}
\item Dieser letzte Abschnitt hat gezeigt, dass...
\item Diese Geschichte kennst du bestimmt noch nicht. 
\item  Die Kinderarmut steigt. Diese Entwicklung muss gestoppt werden. 
\end{enumerate}

\noindent 
\fett{Kennzeichnung der Koreferenz:} In der Spalte \object{Koreferenz} wird die ID zum ersten Token, dass die koreferente Einheit einleitet, angebeben. 


\subsection{Anamnestischer Bezug}

Die NP ist koreferent mit einem mittelbar zuvor genannten Referenten (gemeinsames Wissen über früheren Diskurs wird \hervor{angezapft}), z.B.:

\begin{enumerate}
\item Und dieser/der Prophet, der schon oben erwähnt wurde, sagte....  
\item Er erzählte diese Geschichte mit dem Wolf.... 
\end{enumerate}

\noindent 
\fett{Hinweis zur Identifikation:}
\begin{itemize}
\item Es existiert ein koreferentes Konzept in mittelbarer Diskursumgebung (d.h. 4-n Sätze zuvor). Die Koreferenz muss gekennzeichnet werden.
\item Identifizierende Attribute (z.B. Adjektive, Relativsätze) können ein Hinweis für diesen Gebrauchskontext sein.
\item NP wird mit der Wiedererwähnung zum Topik, also grob ausgedrückt, das, worum es im weiteren Diskursverlauf geht. 
\end{itemize}

\noindent 
\fett{Kennzeichnung der Koreferenz:} In der Spalte \object{Koreferenz} wird die ID zum Kopf der koferenten NP angegeben.

\subsection{Abstrakt-situativer Bezug}

Der Referent gilt in einem bestimmten Bezugsrahmen als einzigartig und damit als eindeutig identifizierbar, z.B.:

\begin{enumerate}
\item Die Braut hatte goldene Schuhe an.
\item Ich war eben bei der Post.  
\end{enumerate}

\noindent 
\fett{Hinweis zur Identifikation:} Referent ist ohne Vorerwähnung alleine durch das Weltwissen eindeutig identifizierbar. (Test: Substitution durch ein Demonstrativum nicht möglich, keine Betonung des Artikelwortes). 

\subsection{Assoziativ-anaphorischer Bezug}

Der Referent wird in einem bestimmten Bezugsrahmen (Frame) kognitiv aktiviert und ist deswegen eindeutig identifizierbar, z.B.:

\begin{enumerate}
\item Beim Fußballspiel ist der Ball kaputt gegangen. 
\item Gestern gab es einen Anschlag. Die Täter sind noch auf der Flucht
\end{enumerate}

\noindent 
\fett{Hinweis zur Identifikation:} Referent ist ohne Vorerwähnung alleine durch das Weltwissen eindeutig identifizierbar. Es gibt immer ein Bezugselement, welches mit einem daran anknüpfende definite Ausdruck in einem logischen oder kulturellen Assoziationsverhältnis steht. (Test: Substitution durch ein Demonstrativum nicht möglich, keine Betonung des Artikelwortes). 

\subsection{Monoreferenz}

Wenn eine NP genau einen Referenten hat, liegt Monoreferenz vor. Es gibt zwei Hauptgruppen von Nomina mit Monoreferenz: Die Eigennamen (z.B. \object{David}) und die Unika (z.B.\object{Sonne}). Die Eigennamen sind in den Daten schon als solche markiert sind. 

\begin{enumerate}
\item Heute scheint die Sonne.
\item Monika hat heute Geburtstag.
\end{enumerate}

\noindent 
\fett{Hinweis zur Identifikation:} Referenten, die in der Wissensgemeinschaft genau \object{einmal} vorkommen, z.B. \object{Gott, Teufel, Erlöser, Sonne, Mond}. 


\subsection{Generische Lesart}

Eine generische Lesart liegt vor, wenn nicht auf ein einzelnes Individuum, sondern eine ganze Gattung Bezug genommen wird. Die NP referiert also nicht, sondern beschreibt lediglich, z.B.: 

\begin{itemize}
\item Die Menschen sind Säugetiere.
\item Eine Katze hat vier Beine.
\end{itemize}

\noindent 
\fett{Hinweis zur Identifikation:} Es liegt eine allgemeine Aussage (Prädikation) über eine bestimmte Gattung vor, die potentiell für alle Vertreter der Gattung gilt. In den biblischen Texten finden sich oftmals Verallgemeinerungen über \object{die Juden} oder die \object{(Un-)Gläubigen}, aber auch Gleichnisse \object{Der gute Baum trägt gute Früchte}. 

\subsection{Nicht-spezifischer Referent}

Es wird nicht auf einen konkreten, identifizierbaren Referenten Bezug genommen, z.B.: 

\begin{enumerate}
\item Sie geht in die Kirche. 
\item zwischen den Zeilen lesen
\item mit letzter Kraft
\item Er ist (ein) Lehrer.
\end{enumerate}

\noindent
\fett{Hinweis zur Identifikation:} Nicht-Spezifische Referenten kommen in Prädikativen, Funktionsverbgefügen oder Adverbialen vor (z.B. \object{séragemo múate} (Otfrid) \hervor{mit traurigem Gemüt}). Oft sind es Massennomen oder nicht zählbare Abstrakta. Aber auch Konkreta als Teil einer allgemeinen Handlung sind möglich. 

\subsection{Spezifisch-indefiniter Referent}

Der Verweis der NP gilt einem salienten oder für den Diskurs besonderen Referenten, der dem Sprecher bekannt ist, der aber nicht näher bestimmt wird, z.B.: 

\begin{enumerate}
\item Er war ein (bestimmter) König...
\item Gesucht wird eine Hilfskraft, die Niederdeutsch sprechen kann.
\item Neulich wurde sie von so einem Verkäufer angerufen.
\end{enumerate}

\noindent 
\fett{Hinweis zur Identifikation:} Phrasen, die im Ahd. mit \object{ein} (PoS-Tag = DI, nicht CARD) oder \object{sum} eingeleitet werden. 

\subsection{Existentielle Indefinita}

Die NP bezieht sich auf einen (potentiell konkreten) Referenten, der nicht näher spezifiziert wird und neu in den Diskurs eingeführt wird. Der Referent ist sowohl für den Sprecher als auch für den Hörer unbekannt, z.B.:

\begin{enumerate}
\item Dort stehen viele/einige/drei Menschen.
\item Ein Bote hat das Paket gebracht.  
\end{enumerate}

\noindent 
\fett{Hinweis zur Identifikation:} Phrasen, die im Ahd. von DI-Token (z.B. \object{manig}) oder Kardinalzahlen begleitet werden. Auch nackte NPs oder NPs in negierten Kontexten fallen in diese Kategorie sowie NPs die in Kombination mit einem stark flektierten Adjektiv stehen. 

\subsection{Possessiver Bezug}

Die NP trägt einen Possessivartikel bei sich und wird damit identifizierbar gemacht, z.B.:

\begin{itemize}
\item Deine Wohnung ist schön.
\item Unsere Annotationen stimmen überein.
\end{itemize}

\noindent 
\fett{Hinweis zur Identifikation:} Phrasen, die im Ahd. von DPOS-Token (z.B. \object{sin}) eingeleitet werden.

\subsection{Ambige Fälle}

Zu dieser Kategorie zählen Fälle, die nicht eindeutig in \object{eine} Kategorie eingeordnet werden können. Ambige Fälle gibt es z.B. wenn der Referent sowohl unmittelbar vorerwähnt, als auch über das Weltwissen als identifizierbar gekennzeichnet werden kann (= \object{ambig\_anaphorisch\_abstrakt}). (Diese können dann als mögliche Brückenkontexte gelten.)

\section{Annotation der Nominalphrase}\label{sec:richtlinien-np}

Nachfolgend wird dokumentiert wie und mit welchen Variablen die Nominalphrase annotiert wird. 

\subsection{Informationen zur Stichprobe}

Die Variable \object{stichprobe.NA} enthält entweder TRUE (NA in Stichprobe vorhanden) oder FALSE (NA nicht in Stichprobe vorhanden). Sie bildet den Ausgangspunkt für die NP-Annotation.

Die Spalte \object{kopf} erhält die Annotation \hervor{kopf}, wenn das durch \hervor{stichprobe.NA} = \hervor{TRUE} ausgewählte Substantiv Kopf einer unmittelbaren Konstituente im Satz ist (= Satzglied); Ebenso: Substantiv ist Teil einer PP als Satzglied. Falls dies nicht der Fall ist, wird die ID des Kopfes annotiert und der entsprechende Kopf mit \hervor{kopf} gekennzeichnet. Der Kopf bildet den Ausgangspunkt für die weitere Annotation der NP. 


\subsection{Lateinische Vorlage}

Die Spalte \object{lat.vorlage} enthält die lateinischen Entsprechung zum Ahd.
Die Token entstammen beim Isidor dem lateinischen Text des Referenzkorpus Altdeutsch. Beim Tatian stammt der lateinische Text aus der Titus-Textdatenbank (\url{http://titus.uni-frankfurt.de/texte/etcs/germ/ahd/tatian/tatialex.htm}, zuletzt aufgerufen am 20.02.2020). 

In der Spalte \object{latein} wird dokumentiert, ob es Abweichungen in der lateinischen Vorlage im Vergleich zur ahd. Übersetzung gibt und wenn ja, um was für eine Abweichung es sich handelt. Wenn also bspw. in der Spalte ein \object{nach} annotiert ist, dann bedeutet dies, dass das Token im Lateinischen dem Kopf nachgestellt ist. Es werden nur Abweichung in Bezug auf die NP annotiert (Stellung der Phrase im Satz gilt nicht als Abweichung), s. Tabelle \ref{tab:lat-abweichung}}. Wenn keine Abweichung vorliegt, wird der Beleg mit \object{keine.abw} annotiert. 


\begin{table}[h!]
\centering
\begin{tabular}{@{}lp{11cm}@{}}
\toprule
\textbf{Annotation}  & \textbf{Erklärung}\\ \midrule
kein.dem             & ein \object{dër} wurde eingefügt, ohne dass ein lateinisches Demonstrativum als Vorlage dient.\\
nach                 & das lateinische Element ist dem Kopf nachgestellt\\
voran                & das lateinische Element ist dem Kopf vorangestellt\\
nach.distanz         & das lateinische Element ist nicht adjazent mit dem Kopf und nachgestellt\\
voran.distanz        & das lateinische Element ist nicht adjazent mit dem Kopf und vorangestellt\\
token.fehlt          & in der Vorlage gibt es kein äquivalentes Token, d.h. der ahd. Schreiber hat das Token ohne Vorlage eingefügt.\\
token.mehr           & in der Vorlage existiert ein Token, das im Ahd. nicht übersetzt worden ist.\\
token.diff\_keine.pp & im Lateinischen gibt es keine PP, die der ahd. PP entspricht. Was in der Vorlage steht, kann der Variablen \hervor{latein.vorlage} entnommen werden.\\
pos\_partizip        & das ahd. N bezieht sich auf ein lateinisches Token, das einer anderen Wortart entspricht; hier ein Partizip.\\
flex\_akkusativ      & das ahd. N bezieht sich auf ein lateinische Token, das eine andere Flexion hat; hier: Token steht im Akkusativ.\\
syn.fkt\_subjekt     & das ahd. N bezieht sich auf ein lateinische Token, das eine andere syntaktische Funktion hat; hier: Subjekt\\\bottomrule
\end{tabular}
\caption{Annotationsmöglichkeiten Differenzbelege (Spalte \object{lat.abweichung})}
\label{tab:lat-abweichung}
\end{table}

Kombinationen von mehreren Abweichungen, z.B. Stellung +  Wortart werden mit Unterstrich kombiniert, z.B. \hervor{nach\_adjektiv}.

\subsection{Verweis auf Bibelstelle}

Der Verweis auf die Bibelstelle wird nur bei Tatian gemacht und dient in erster Linie als Übersetzungshilfe. Die Spalte \object{bibel} zeigt die Bibelstelle an (etwa Mt 1,5). Die Spalte \object{trans.luther1912} gibt einen Übersetzungsvorschlag von \url{http://www.bibel-online.net/buch/luther_1912/} (zuletzt aufgerufen am 10.02.2017) an.  

\subsection{Informationen zum Kopf der NP}

Die Spalte \object{phrase} enthält die Information, ob der Kopf nur Teil einer NP (= \hervor{np}) ist oder innerhalb einer NP in eine PP eingebettet ist. Wenn es sich um einen PP handelt, wird zusätzlich die Art der Pröposition angegeben (z.B. \object{pp\_mit}). 


\subsection{Ergänzungen der NP}

Ergänzungen der NP (Determinierer, Attribute) werden einzeln annotiert, um leichter Kombinationen abfragen zu können. Über die Angabe der ID lässt sich die Reihenfolge der Elemente innerhalb der NP abrufen (> bedeutet Nachstellung, < bedeutet Voranstellung). 

\begin{table}[h!]
\centering

\begin{tabular}{@{}lp{11cm}@{}}
\toprule
\textbf{Variable} & \textbf{Ausprägungen}\\\midrule
id.ther           & Angabe der ID von \object{dër}; wenn nicht vorhanden: nein\\
id.np.attr        & Angabe der ID von nominalen Attributen (meist sind es Genitivattribute); wenn nicht vorhanden: nein\\
id.adj.attr       & Angabe der ID von Adjektivattributen; wenn nicht vorhanden: nein\\
id.poss           & Angabe der ID von Possessivartikeln; wenn nicht vorhanden: nein\\
id.zahl           & Angabe der ID von modifizierenden Zahlwörtern (CARD); wenn nicht vorhanden: nein\\
id.demonstr       & Angabe der ID von zusammengestetzen Demonstrativartikeln (dëser); 
 wenn nicht vorhanden: \object{nein}\\
id.andere.mod     & Angabe der ID von weiteren Modifizierern des Kopfes, z.B. Fragewörtern, demonstratives sëlb- oder \object{sum}; wenn nicht vorhanden: nein\\
id.rel.pro.nach   & Angabe der ID von Relativpronomen, die einen attributiven Relativsatz (zur NP) einleiten;  wenn nicht vorhanden: nein\\\bottomrule
\end{tabular}
\caption{Annotationsmöglichkeiten (NP-Struktur)}
\end{table}

\subsection{Anmerkungen} 
In der Spalte \object{anmerkungen} ist Raum für jegliche Anmerkungen und Auffälligkeiten zum Beleg. Auch Bezüge zur Forschungsliteratur sind hier vermerkt.

\subsection{Korrekturen im DDDTS} 

In der Spalte \object{korrekturen.dddts} werden Korrekturen in der Annotation dokumentiert. Dies ist nur notwendig, wenn die eigene Analyse der NP maßgeblich von der grammatischen Auszeichnung, die im Projekt vorgenommen wurde, abweicht. 

\subsection{Übersetzungsprobleme} 
Die Spalte \object{problematisch.trans} dient dazu, problematische Belege oder Unklarheiten bei der Übersetzung zu dokumentieren. 
  
\subsection{Definitheit und Koreferenz} 

In der Spalte \object{np.definit} werden Informationen zur (In-)Definitheit des jeweiligen Referenten gemacht (s. ausführlich Abschnitt \ref{sec:richtlinien-definitheit}). Bei Vorerwähnung wird in der Spalte \object{koreferenz} die ID der Bezugs-NP angegeben. Wenn es keine Vorerwähnung gibt, dann steht hier ein \object{nein}.


\section{Syntaktische Funktion und semantische Rolle} \label{sec:richtlinien-semantische-rolle}

Die Spalte \object{synt.fkt} gibt (im Rahmen der Annotation der NP) an, welche syntaktische Funktion die jeweilige NP übernimmt. Unterschieden werden folgende Kategorien: Subjekt (= \object{subjekt}), Objekt (= \object{objekt}), Adverbial (= \object{adverbial})%\footnote{Es wird nicht zwischen fakultativen und obligatorischen Adverbialen unterschieden.}
, Attribut (= \object{attribut}), Prädikativ (= \object{praedikativ}), nominaler Teil des Prädikates bei Funktionsvergefügen (= \object{nom.teil.praedikat}), direkte Anrede (= \object{anrede}. Appositionen werden nach folgendem Muster gekennzeichnet: \object{subjekt.apposition} oder \object{object.apposition} etc. 

In der Spalte \object{sem.rolle} wird zwischen \object{agens} und \object{nicht-agens} differenziert. Die Agensrolle wird nur bei Subjekten vergeben, die willentlich handeln oder ein Handlung verursachen, eine selbstinduzierte Bewegung ausführen oder Träger eines Gefühlszustands sind \parencite[gemäß den Proto-Agensdimensionen in][]{Dowty1991,Primus2012}. Ist die Zuordnung unklar, wird immer \object{nicht-agens} annotiert. 
 
