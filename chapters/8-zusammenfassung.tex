\chapter{Zusammenfassung und Ausblick}\label{kapitel:zusammenfassung}

Die vorliegende Arbeit hat die Konstruktionalisierung von [\object{dër} + N] aus drei Blickwinkeln betrachtet. Erstens wurde die semantische Ausbleichung des ursprünglichen Demonstrativartikels beleuchtet. Zweitens wurde gezeigt, mit welchen Substantivklassen der N-Slot besetzt wird. Drittens wurden Strukturmerkmale der gesamten Nominalphrase offengelegt. 
Diese Herangehensweise trägt der Tatsache Rechnung, dass die Entwicklung des Definitartikels nicht nur auf Morphemebene abläuft (aus einem demonstrativen adnominalen Element wird ein auf Definitheit reduziertes Artikelwort), sondern die gesamte Nominalphrase im Althochdeutschen betrifft. Dieser gesamtheitliche Blick steht im Einklang mit der Grundidee der Konstruktionsgrammatik, dass Sprache als dynamisches Netzwerk von miteinander assoziierten Konstruktionen betrachtet wird. Sie nimmt eine kognitive Perspektive auf Grammatik (und Sprache im Allgemeinen) ein, weil davon ausgegangen wird, dass kognitive Grundprinzipien (vor allem Ka"-te"-gori"-sier"ungs- und Ab"-strak"-tions"-prozesse) das Sprachsystem formen. In der vorliegenden Arbeit konnte gezeigt werden, wie der kognitiv-linguistische Faktor Belebtheit, d.h. die außersprachliche Einordnung von Entitäten in \textsc{menschlich, belebt, unbelebt} die Setzung von \object{dër} beeinflusst. Auch Entrenchmentprozesse, welche die Fähigkeit zu abstrahieren und analogische Bezüge herzustellen voraussetzen, wurden erstmals mit der Entwicklung des Definitartikels in Verbindung gebracht.

Im Gegensatz zu bisherigen Studien zum Definitartikel beruhen die Analysen auf einer breiten Datenbasis, die mithilfe korpuslinguistischer Methoden untersucht wurden. Als Textgrundlage dienten die fünf größten ahd. Textdenkmäler, die über das \object{Referenzkorpus Altdeutsch} zugänglich sind: Isidor (um 790), Monseer Matthäus (um 810), Tatian (um 840), Otfrids Evangelienbuch (um 870) und Notkers Boethius (um 1025). 

Die Funktionsanalyse von \object{dër} hat gezeigt, dass die Entwicklung des Definitartikels schon im frühen Althochdeutschen weit fortgeschritten ist. Bereits im Isidor finden sich in 100 zufällig ausgewählten NPs mehr \object{dër}-Belege in semantisch-definiten, d.h. situationsunabhängigen Kontexten, als in pragmatisch-definiten, d.h. situationsabhängigen Kontexten. Gemessen an den später datierten Texten steigt diese Kontextexpansion: Während im Tatian knapp ein Drittel der se\-man\-tisch-definiten Fälle mit einer \object{dër}-Phrase ausgedrückt werden (Unika und generische Belege ausgenommen), sind es bei Otfrid schon die Hälfte. Auch die Ergebnisse zu den inhärent definiten Superlativkonstruktionen sowie den Unika spiegeln diese funktionale Verschiebung. Bei Notker, dem jüngsten Text, dominiert in diesen Kontexten die \object{dër}-Setzung. Es konnte gezeigt werden, dass sowohl der anaphorische als auch der anamnestische Gebrauch als mögliche Brückenkontexte und damit als \hervor{Startpunkte} für die Entwicklung in Frage kommen. Der generische Gebrauch ist früher möglich, als es die in der Forschung bisher vorgeschlagen Grammatikalisierungsskalen postulieren, weshalb der Hauptentwicklungspfad, der eine Expansion von pragmatisch-definiten Kontexten zu semantisch-definiten Gebrauchskontexten vorsieht, um einen generischen \hervor{Seitenpfad} erweitert wurde. Es bleibt zukünftigen Studien überlassen, die Faktoren offenzulegen, die für den variablen Gebrauch von \object{dër} bei generischen Ausdrücken verantwortlich sind. Möglicherweise spielt die Art des generischen Verweises eine Rolle (\object{kind-referring NP} mit und ohne \object{characterizing sentences}, vgl. Abschnitt \ref{sec:nicht-referentiell}). 

Ob ein Substantiv mit \object{dër} kombiniert wird, ist von seinem Belebtheitsgrad abhängig. Grob zusammengefasst, werden in den ahd. Texten insbesondere Menschen, aber auch Konkreta eher determiniert als Abstrakta und Massennomen. Die Unterschiede zwischen den Texten zeigen, dass \object{dër} mit zunehmender Obligatorisierung und semantischer Ausbleichung entlang der Belebtheitshierarchie auf neue Substantivklassen expandiert. Während im Isidor der Belebtheitsfaktor nicht sichtbar wird, zeigen die Auswertungen zum Monseer Matthäus, dass menschliche und konkrete Referenten einen größeren Anteil innerhalb der \object{dër}-Belege einnehmen als bei den Belegen ohne \object{dër}. Im Tatian stechen zwei Gruppen signifikant heraus: Zum einen Menschen, welche überzufällig häufig determiniert werden und zum anderen Abstrakta, die überzufällig häufig undeterminiert bleiben. Betrachtet man nur die Hapax Legomena, so nehmen menschliche Referenten bei Otfrid eine ähnliche Sonderrolle ein, da sie stärker als alle anderen Substantivtypen zur \object{dër}-Setzung neigen. Diese Präferenz hängt damit zusammen, dass Menschen kognitiv auffällig und maximal handlungsfähig sind. Sie sind besonders wichtig für den Diskurs, weswegen Sprecherinnen und Sprecher sie sprachlich hervorheben \hervor{wollen}. Mit seiner ursprünglich demonstrativen Funktion ist \object{dër} hierfür prädestiniert.  
Bei Notker hat die Belebtheitsanalyse gezeigt, dass die  \object{dër}-Präferenz für Menschen, Tiere und Konkreta gleichermaßen hoch ist. Anders als in den älteren Texten ist die \hervor{Abneigung} der Abstrakta gegenüber der Determinierung jedoch nicht mehr so stark. Der Faktor Relevanz ist auf unterschiedliche Weise sichtbar geworden. So sind es im Monseer Matthäus und im Tatian vor allem gesellschaftlich ranghohe und männliche Referenten, die regelmäßig determiniert werden, so dass hier Relevanz mit Belebtheit positiv korreliert. Bei Otfrid werden mithilfe von \object{dër} thematisch wichtige Referenten hervorgehoben, darunter auch viele Konkreta. Im Isidor und auch bei Notker scheinen viele Abstrakta auch thematisch relevant zu sein, so dass dies das relativ hohe Vorkommen von \object{dër} erklären könnte. Hier müssten zukünftig noch weitere textuelle \hervor{Tiefbohrungen} erfolgen.

Die Anfertigung von Annotationsrichtlinien sowie die doppelte Annotation, welche über \herkur{Inter Annotator Agreements} evaluiert wurde, hat für ein hohes Maß an Transparenz und Objektivität bei der Belebtheitsannotation gesorgt. In zukünftigen Untersuchungen könnten mit einer ähnlichen methodischen Herangehensweise auch die semantischen Rollen untersucht werden. Die Ergebnisse der Stichprobenanalysen zum Isidor, Tatian und Otfrid deuten zwar darauf hin, dass Agentivität die \object{dër}-Setzung begünstigt, allerdings sind agentive Referenten meist auch belebt. Um die Wechselwirkung zwischen Belebtheit und semantischer Rolle offenzulegen, müsste die semantische Rolle noch feiner ausdifferenziert und dann systematisch auf die ahd. Texte übertragen werden. 

Die Ergebnisse zur Struktur der Nominalphrase haben sichtbar gemacht, dass schon ab dem frühen Althochdeutschen in der Nominalphrase ein pränominaler Determiniererslot angelegt ist. In allen Textdenkmälern wird der Großteil aller definiten Phrasen von einem Determinierer ähnlichen Element (neben \object{dër} vor allem Possessivartikel, aber auch das Demonstrativum \object{dëser} oder Genitivattribute) begleitet. Die flektierbaren Einleiter sind darüber hinaus auch noch außerordentlich stellungsfest. Es ist wahrscheinlich, dass Sprecherinnen und Sprecher aus diesem empirischen Input eine Art Determiniererschema ableiten, in dem \object{dër} aufgrund seiner hohen Gebrauchsfrequenz den Determiniererslot standardmäßig besetzt, was die Herausbildung der Definitartikelkonstruktion [\object{dër} + N] fördert. Zudem begünstigt das Schema [\object{dër} + Adjektiv\textsubscript{schwach} + N] als Resultat semantisch bedingter Kollokationen die Obligatorisierung von \object{dër}. Auch spezifische, hochfrequente Konstruktionen wie \object{dër heilant} sind für den Wandel förderlich. Sie treten primär in semantischen Definitheitskontexten auf und können damit als erste Instanzen des Schemas [Definitartikel + Nomen] analysiert werden. In dieser Funktion dienen sie als Vorbild für die analogische Ausbreitung der Konstruktion. 






