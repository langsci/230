\chapter{Einleitung}

Wie in vielen anderen Sprachen hat sich der deutsche Definitartikel aus einem adnominal gebrauchten Demonstrativum entwickelt  \parencite{Oubouzar1992,Szczepaniak2011a}. Die entscheidende Phase des Wandels spielt sich im Althochdeutschen (750-1025) ab. Das ursprüngliche Demonstrativum \object{dër}\footnote{Im Folgenden wird die normalisierte Form \object{dër} verwendet, um auf das ahd. System Bezug zu nehmen. Alle nachfolgenden ahd. Beispiele werden -- wenn nicht anders angegeben -- in der überlieferten Form dargestellt.} verliert in dieser Zeit seine demonstrative, d.h. verweisende Funktion und erschließt Gebrauchskontexte, in denen die eindeutige Identifizierbarkeit des Referenten auch unabhängig von der Gesprächssituation gewährleistet ist. Obwohl diese Entwicklung durch die ahd. Überlieferungslage gut dokumentiert ist, fehlen bislang systematische Korpusuntersuchungen. Mit \textcite{Hodler1954,Oubouzar1989,Oubouzar1992,Oubouzar1997a} liegen zwar durchaus größere empirische Arbeiten vor. Allerdings liegen diesen keine klaren semantischen Analysekriterien zugrunde. Auch Brückenkontexte \parencite{Heine2002a}, die \textcite{Himmelmann1997} für außereuropäische Sprachen ausgearbeitet hat, sind bisher ununtersucht geblieben. Die Studien von \textcite{Abraham1997,Philippi1997,Leiss2000}, die mögliche Antworten auf die Frage liefern, warum sich der Definitartikel überhaupt entwickelt hat, stützen sich auf Analysen von Einzelbeispielen. Das Gleiche gilt für die Untersuchungen von \textcite{Demske2001,Kraiss2012,Kraiss2014,Schlachter2015}, welche den funktionalen Wandel in den Fokus rücken.
In der vorliegenden Arbeit wird die Entwicklung des Definitartikels erstmals an einer größeren Datenmenge computergestützt und mit korpuslinguistischen Methoden untersucht. Ermöglicht wird dieses Vorgehen durch die zunehmende Digitalisierung und Annotation historischer Daten der letzten Jahre. Die Textgrundlage für die Korpusuntersuchung bilden die fünf größten ahd. Textdenkmäler aus dem  \object{Referenzkorpus Altdeutsch} \parencite{Donhauser2015}.

\section{Zielsetzung} 
%wieviel verraten? Ergebnisse
Mit der Untersuchung soll der funktionale Wandel des Definitartikels empirisch erschlossen werden. Das Ziel ist, über systematische Analysen von Gebrauchskontexten, in denen \object{dër} erscheint, die ersten Stufen des Grammatikalisierungspfades \parencite{Lehmann2015} für das Deutsche zu rekonstruieren und auf den Prüfstand zu stellen. Es wird davon ausgegangen, dass zu den ursprünglich prag"-ma"-tisch-definiten (situationsabhängigen) Gebrauchskontexten se"-man"-tisch-definite (situationsunabhängige) hinzukommen \parencite{Lobner1985,Himmelmann1997}. 
Die Entwicklung des Definitartikels spielt sich aber nicht nur auf  Morphem-, sondern auch auf Phrasenebene ab, da sich erst in Verbindung mit einem Substantiv aus dem ursprünglichen Demonstrativ- ein Definitartikel entwickeln konnte. In anderen Kontexten grammatikalisieren Demonstrativa bspw. zu Konjunktionen oder Personalpronomen \parencite{Diessel1999}. Aus der bisherigen Forschung \parencite[u.a.][]{Oubouzar1989,Oubouzar1992}
lässt sich ableiten, dass nicht alle Substantivtypen gleichermaßen mit \object{dër} kombiniert werden. Aufbauend auf \textcite{Szczepaniak2011a,Enger2011} wird die Hypothese aufgestellt, dass der kognitiv-linguistische Faktor Belebtheit (in Interaktion mit Individualität, Relevanz und Agentivität) bestimmt, welche Substantive determiniert werden. Es wird davon ausgegangen, dass die Expansion bei menschlichen und kommunikativ-relevanten Referenten beginnt und weiter entlang der Belebtheitsskala in Richtung Abstrakta und Massennomen verläuft. Der Zusammenhang von Belebtheit und \object{dër}-Gebrauch soll mit einer Korpusuntersuchung empirisch nachgewiesen werden.  

Die Untersuchung ist im Rahmen der historischen Konstruktionsgrammatik verankert \parencite[s. u.a.][]{Traugott2003,Bergs2008,Traugott2013}. Die zentrale Idee der Konstruktionsgrammatik ist, dass Sprache aus einem stukturierten und über die Zeit veränderbaren Netzwerk von konventionalisierten Form-Funktionspaaren, den Konstruktionen, besteht und sich gebrauchsbasiert wandelt \parencite{Bybee2010,Bybee2013}. Die Entwicklung des Definitartikels wird dabei als Konstruktionalisierung aufgefasst. Vor diesem theoretischen Hintergrund wird auch der Frage nachgegangen werden, inwiefern form- oder funktionsseitig ähnliche Nominalphrasen den Wandel von [\object{dër} + N] analogisch beeinflussen. Zentral ist dabei die Idee, dass sich \object{dër} als \object{Default}-Marker für Definitheit innerhalb eines Determinierschemas etabliert, welches Sprecherinnen und Sprecher auf Basis ähnlicher adnominaler Definitheitsmarker, etwa  [Possessivartikel + N]  oder [Genitivattribut + N] abstrahieren \parencite[vgl. fürs Englische][]{Sommerer2015}.


%\begin{figure}
%\begin{center}
%  \includegraphics[width=10 cm]{images/variablen-konst.jpg}
%\caption {Variablen der Konstruktionalisierung}
%\label{abb:variablen}
%\end{center}
%\end{figure} 


%ein abstraktes Determinierer-Schema herausbildet, indem sich \object{dër} als Default-Marker für Definitheit etabliert \parencite[ähnlich][]{Sommerer2015}. 


\section{Aufbau} 

Der Hauptteil der Arbeit besteht aus vier Theoriekapiteln  (Kapitel 2-5), einem Methodenteil (Kapitel 6) und einem Ergebnisteil, der sich aus der Ergebnispräsentation (Kapitel 7) und der theoretischen Diskussion der Ergebnisse (Kapitel 8) zusammensetzt. Der letzte Teil (Kapitel 9) fasst die zentralen Ergebnisse zusammen und weist auf offene Fragen hin sowie mögliche Anknüpfungspunkte, die sich aus den Ergebnissen der Arbeit ergeben. 

Im zweiten Kapitel erfolgt zunächst eine Auseinandersetzung mit den Grammatikalisierungspfaden von \textcite{Greenberg1978} und \textcite{Lehmann2015}, die die wichtigsten funktionalen und formalen Schritte des Wandels vom ursprünglichen Demonstrativ- zum Definitartikel abbilden. Anschließend wird die Quelle dieser Grammatikalisierung genauer betrachtet, da im Gegensatz zu typischen Grammatikalisierungsprozessen kein lexikalisches, sondern ein grammatisches Element am Anfang der Entwicklung steht. Nachdem im nachfolgenden Abschnitt die wichtigsten Parameter der Grammatikalisierung zusammengetragen werden, nimmt der zweite Teil des Kapitels eine konstruktionsgrammatische Perspektive ein. Es wird dafür argumentiert, dass es sich bei der Entwicklung des Definitartikels um einen Fall von Konstruktionalisierung handelt, genauer um die Herausbildung des Schemas [Definitartikel + N]. Angetrieben wird dieser Wandel durch Type- und Tokenentrenchment sowie Analogie- und Reanalysemechanismen. 

Das dritte Kapitel fasst die wichtigsten Erkenntnisse aus der Forschung zur Entwicklung des Definitartikels im Althochdeutschen zusammen. Nach einem Überblick über die Strategien, die das Althochdeutsche kannte, um Definitheit auszudrücken, steht die Frage nach dem Warum im Mittelpunkt. Zunächst werden die klassischen Gründe, die die Forschung für die Herausbildung des Artikels anführt, kritisch beleuchtet. Anschließend wird ein weiteres Entstehungsszenario vorgestellt, das den Faktor Expressivität mit ins Spiel bringt: Es wird davon ausgegangen, dass der Demonstrativartikel ursprünglich dazu diente, diskurswichtige Referenten zu exponieren. Der inflationäre Gebrauch dieser Strategie hat den funktionalen Wandel und die Konventionalisierung von [\object{dër} + N] angetrieben. Das Kapitel schließt mit einer Übersicht über Faktoren, die den Wandel determinieren sowie einer Diskussion möglicher Brückenkontexte. 

Im vierten Kapitel werden Kriterien zusammengetragen, mit denen sich De"-mon"-stra"-tiv- von Definitartikeln funktionsseitig voneinander abgrenzen lassen. Zu Beginn werden die wichtigsten Theorien aus der Definitheitsforschung  skizziert. Damit die funktionale Reichweite von Demonstrativartikeln sichtbar wird, werden im nächsten Teil Gebrauchskontexte erläutert, in denen Demonstrativartikel typischerweise vorkommen. Danach stehen Kontexte, die  Definitartikeln vorbehalten sind, im Fokus. 
Im Rahmen der Löbnerschen Definitheitstheorie \parencite{Lobner1985}  werden die für Definitartikel genannten Gebrauchskontexte als sogenannte semantische Definitheitskontexte beschrieben -- in Abgrenzung zu  pragmatischen Definitheitskontexten, die zur Domäne der Demonstrativartikel gehören. Aus der bisherigen Forschung lässt sich ableiten, dass das ahd. \object{dër} in beiden Kontexten auftreten kann. Mit der Korpusuntersuchung soll das Gebrauchsspektrum erstmals systematisch an einer größeren Datenmenge untersucht werden. 

Im fünften Kapitel wird dafür argumentiert, dass die Entwicklung des Definitartikels eine Form von belebtheitsgesteuertem Wandel darstellt. Zu Beginn wird die kognitiv-linguistische Kategorie Belebtheit, welche sich über unterschiedliche Belebtheitshierarchien abbilden lässt, definiert. Anschließend wird mithilfe bisheriger Erkenntnisse aus der historischen Sprachwissenschaft, der Sprachtypologie sowie der kognitiven Linguistik gezeigt, warum es sinnvoll ist, Belebtheit als Einflussfaktor hinzuzuziehen. Das hier theoretisch modellierte Belebtheitsmodell umfasst auch Abstrakta und Massennomen und bezieht damit auch den Faktor Individualität mit ein. Außerdem werden Korrelationen von Belebtheit und Relevanz sowie semantischer Rolle aufgezeigt.

Das sechste Kapitel widmet sich der Untersuchungsmethode. Am Anfang wird diskutiert, inwiefern sich das Althochdeutsche mit korpuslinguistischen Methoden erschließen lässt. Danach wird die Textauswahl begründet. Um eine vollständige Transparenz zu gewährleisten, werden die Schritte der Datenaufbereitung genau erläutert. Eine besondere methodische Herausforderung ist die Annotation von Belebtheitheitskategorien: Auf Basis der Übersetzungen aus dem \object{Referenzkorpus Altdeutsch} wurden Konzept-Types generiert, welche mit Hilfe von Annotationsrichtlinien doppelt annotiert und über \object{Inter Annotatator Agreements}  evaluiert wurden. Außerdem wurde eine Stichprobe an Nominalphrasen nach Definitheitskontexten, semantischer Rolle sowie morphosyntaktischen Merkmalen klassifiziert. Auch Differenzbelege (Abweichungen von der lat. Vorlage, die bei einigen Texten relevant sind) wurden gekennzeichnet. Der letzte Abschnitt des Kapitels zeigt, wie quantitative und qualitative Analysemethoden bei der Untersuchung kombiniert werden. 
  
Im siebten Kapitel erfolgt die Präsentation der Ergebnisse. Zu Beginn wird zunächst ein Überblick gegeben, wie häufig [\object{dër} + N] in den einzelnen Texten vorkommt. Anschließend wird das funktionale Spektrum der Konstruktion beleuchtet. Hierzu wird die Distribution von pragmatischen und semantischen Definitheitskontexten, in denen \object{dër} erscheint, präsentiert. Anschließend werden die Häufigkeiten von Superlativkonstruktionen sowie ausgewählte Unika mit und ohne Determinierung gegenüberstellt; beide Kontexte repräsentieren semantische Definitheit. Der dritte Abschnitt des Kapitels enthält die Ergebnisse zu den kognitiven Faktoren Belebtheit, Individualität, Relevanz und semantische Rollen. Im Anschluss wird die NP-Perspektive eingenommen und gezeigt, wie salient \object{dër} als Phraseneinleiter in den einzelnen Texten ist. Auch das Stellungsverhalten einzelner Determinierer und attributiver Adjektive sowie Interaktionen von \object{dër} mit schwach flektierten Adjektiven wird hier offengelegt. 

Das achte Kapitel stellt die Ergebnisse in einen größeren theoretischen Zusammenhang. Wann vollzieht \object{dër} den funktionalen Wandel zum Definitartikel und welche Brückenkontexte dienen als Sprungbrett? Die empirischen Erkenntnisse werden genutzt, um den Entwicklungspfad des Definitartikels neu zu modellieren. Zudem wird ein Expansionspfad entlang der Belebtheitshierarchie vorgeschlagen. Die Ergebnisse zur Struktur der Nominalphrase  stützen die Annahme, dass die Konstruktionalisierung von [\object{dër}+ N] als Teil einer übergeordneten Schematisierung -- dem Determiniererschema -- abgelaufen ist. Ferner wird auf Basis der Ergebnisse gezeigt, wie sowohl Type- als auch Tokenentrenchment den grammatischen Wandel beeinflussen.
